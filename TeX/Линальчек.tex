\documentclass{book}
%nerd stuff here
\pdfminorversion=7
\pdfsuppresswarningpagegroup=1
% Languages support
\usepackage[utf8]{inputenc}
\usepackage[T2A]{fontenc}
\usepackage[english,russian]{babel}
% Some fancy symbols
\usepackage{textcomp}
\usepackage{stmaryrd}
% Math packages
\usepackage{amsmath, amssymb, amsthm, amsfonts, mathrsfs, dsfont, mathtools}
\usepackage{cancel}
% Bold math
\usepackage{bm}
% Resizing
%\usepackage[left=2cm,right=2cm,top=2cm,bottom=2cm]{geometry}
% Optional font for not math-based subjects
%\usepackage{cmbright}

\author{Коченюк Анатолий}
\title{Линейный Анализ}

\usepackage{url}
% Fancier tables and lists
\usepackage{booktabs}
\usepackage{enumitem}
% Don't indent paragraphs, leave some space between them
\usepackage{parskip}
% Hide page number when page is empty
\usepackage{emptypage}
\usepackage{subcaption}
\usepackage{multicol}
\usepackage{xcolor}
% Some shortcuts
\newcommand\N{\ensuremath{\mathbb{N}}}
\newcommand\R{\ensuremath{\mathbb{R}}}
\newcommand\Z{\ensuremath{\mathbb{Z}}}
\renewcommand\O{\ensuremath{\emptyset}}
\newcommand\Q{\ensuremath{\mathbb{Q}}}
\renewcommand\C{\ensuremath{\mathbb{C}}}
\newcommand{\p}[1]{#1^{\prime}}
\newcommand{\pp}[1]{#1^{\prime\prime}}
% Easily typeset systems of equations (French package) [like cases, but it aligns everything]
\usepackage{systeme}
\usepackage{lipsum}
% limits are put below (optional for int)
\let\svlim\lim\def\lim{\svlim\limits}
%\let\svlim\int\def\int{\svlim\limits}
% Command for short corrections
% Usage: 1+1=\correct{3}{2}
\definecolor{correct}{HTML}{009900}
\newcommand\correct[2]{\ensuremath{\:}{\color{red}{#1}}\ensuremath{\to }{\color{correct}{#2}}\ensuremath{\:}}
\newcommand\green[1]{{\color{correct}{#1}}}
% Hide parts
\newcommand\hide[1]{}
% si unitx
\usepackage{siunitx}
\sisetup{locale = FR}
% Environments
% For box around Definition, Theorem, \ldots
\usepackage{mdframed}
\mdfsetup{skipabove=1em,skipbelow=0em}
\theoremstyle{definition}
\newmdtheoremenv[nobreak=true]{definition}{Определение}
\newmdtheoremenv[nobreak=true]{theorem}{Теорема}
\newmdtheoremenv[nobreak=true]{lemma}{Лемма}
\newmdtheoremenv[nobreak=true]{problem}{Задача}
\newmdtheoremenv[nobreak=true]{property}{Свойство}
\newmdtheoremenv[nobreak=true]{statement}{Утверждение}
\newmdtheoremenv[nobreak=true]{corollary}{Следствие}
\newtheorem*{note}{Замечание}
\newtheorem*{example}{Пример}
\renewcommand\qedsymbol{$\blacksquare$}
% Fix some spacing
% http://tex.stackexchange.com/questions/22119/how-can-i-change-the-spacing-before-theorems-with-amsthm
\makeatletter
\def\thm@space@setup{%
  \thm@preskip=\parskip \thm@postskip=0pt
}
\usepackage{xifthen}
\def\testdateparts#1{\dateparts#1\relax}
\def\dateparts#1 #2 #3 #4 #5\relax{
    \marginpar{\small\textsf{\mbox{#1 #2 #3 #5}}}
}

\def\@lecture{}%
\newcommand{\lecture}[3]{
    \ifthenelse{\isempty{#3}}{%
        \def\@lecture{Lecture #1}%
    }{%
        \def\@lecture{Lecture #1: #3}%
    }%
    \subsection*{\@lecture}
    \marginpar{\small\textsf{\mbox{#2}}}
}
% Todonotes and inline notes in fancy boxes
\usepackage{todonotes}
\usepackage{tcolorbox}

% Make boxes breakable
\tcbuselibrary{breakable}
\newenvironment{correction}{\begin{tcolorbox}[
    arc=0mm,
    colback=white,
    colframe=green!60!black,
    title=Correction,
    fonttitle=\sffamily,
    breakable
]}{\end{tcolorbox}}
% These are the fancy headers
\usepackage{fancyhdr}
\pagestyle{fancy}

% LE: left even
% RO: right odd
% CE, CO: center even, center odd
% My name for when I print my lecture notes to use for an open book exam.
% \fancyhead[LE,RO]{Gilles Castel}

\fancyhead[RO,LE]{\@lecture} % Right odd,  Left even
\fancyhead[RE,LO]{}          % Right even, Left odd

\fancyfoot[RO,LE]{\thepage}  % Right odd,something additional 1  Left even
\fancyfoot[RE,LO]{}          % Right even, Left odd
\fancyfoot[C]{\leftmark}     % Center

\usepackage{import}
\usepackage{xifthen}
\usepackage{pdfpages}
\usepackage{transparent}
\newcommand{\incfig}[1]{%
    \def\svgwidth{\columnwidth}
    \import{./figures/}{#1.pdf_tex}
}
\usepackage{tikz}
\begin{document}
    \maketitle
    \section{Введение}
    Трифанов Александр Игоревич

    Два модуля: аналитическая геометрия, линейная алгебра

    Отчётность: дз, кр, лабы, рубежное тестирование, экзамен

    дз (16 штук(8 в модуль) по 2 баллв)
    
    кр (4 (2 в модуль) 5 баллов)

    лаба (1-2 по 5 баллов)

    рубежный тест (1)



    \chapter{I курс}
    \section{Матрицы и операции над ними}

    \begin{definition}
        Матрица -- прямоугольная таблицв чисел
    \end{definition}

    $\begin{bmatrix} 
        a_{11}&a_{12}&\ldots & a_{1n}\\
        a_{21}&a_{22}&\ldots &a_{2n}\\
        \vdots & \vdots & \ldots, \vdots\\
        a_{m 1}& a_{m 2 } & \ldots & a_{m n}
     \end{bmatrix} $

     $a_{i,j}\in \R$ -- элементы матрицы

     $a_{11}, a_{12}, \ldots, a_{1n}$ -- строка 1
     
     $a_{12}, a_{22}, a_{32}, \ldots, a_{m 2}$ -- столбец 2

     $a_{ij}$ -- элемент на пересечении $i$-той строки и $j$-того столбца

     В матрице выше $m$ строк и $n$ столюцов. $A_{m\times n}$ -- обозначение

     \begin{note}
         $n=m \implies A_{n\times n}$ -- квадратная матрица 

         $\{a_{ii}\}_{i=1}^n$ -- диагональ матрицы $A_{n\times n}$
     \end{note}

     \begin{note}
         $A = \| a_{ij} \|$ $B = \| b_{ij} \|$
     \end{note}
    
     \begin{note}
         $A=B \iff $
         \begin{itemize}
             \item одинаковые размеры
             \item $\forall i, j a_{ij} = b_{ij}$
         \end{itemize}
     \end{note}

     Операции с матрицами:
     \begin{enumerate}
         \item Умножение на число:

             $B = \alpha \cdot A \iff  \forall i, j\quad b_{ij} = \alpha \cdot a_{ij}$
         \item Сложение:

             Пусть $A,B$ -- одинакового размера

             $A+B = C :\quad c_{ij} = a_{ij} + b_{ij} \forall i,j$

             \begin{note}
                 $\sphericalangle \mathbb{O}: \mathbb{O} + A = A+ \mathbb{O} = A$

                 $\mathbb{O}$ -- полностью состоит из нулей
             \end{note}
     \end{enumerate}

     Свойства:
     \begin{itemize}
         \item коммутативность сложения (следует из коммутативности сложения чисел) $A+B = B+A$
         \item ассоциативность (--||--) $(A+B)+C = A+(B+C)$
         \item дистрибутивность $\alpha(A+B) = \alpha A + \alpha B$
         \item $\forall A\quad -A = -1\cdot A:\quad A+(-A)= \mathbb{O}$ противоположный элемент по сложению
     \end{itemize}

     \begin{enumerate}
         \item [3] Умножение матриц
             
             Пусть $A_{m\times l}, B_{l\times n}$

             $C = A\cdot B\quad c_{ij} = \sum_{k=1}^{l} a_{ik}\cdot b_{kj}= C_{m\times n}$
     \end{enumerate}
\begin{note}
    $A\cdot B \neq B\cdot A$

    Для квадратных матриц вводится такое понятие, как коммутатор 

    $[AB] = A\cdot B - B\cdot A$
\end{note}

\begin{example}
    $A\cdot D = \begin{bmatrix} 1&2&0\\1&-1&1 \end{bmatrix} \cdot \begin{bmatrix} 1&2&3\\4&5&6\\7&8&9 \end{bmatrix}  = \begin{bmatrix} 9&12&15\\4&5&6 \end{bmatrix} $


\end{example}

    \begin{note}
        $\sphericalangle I: A\cdot I = I\cdot A = A$

        $I = \begin{bmatrix} 1&0&\ldots&0\\
        0&1&\ldots&0\\ 
        \vdots&\vdots&\vdots&\vdots\\
        0&0&\ldots&1 \end{bmatrix} $
    \end{note}

    Свойства:
    \begin{itemize}
        \item некоммутативность $A\cdot B \neq B\cdot A$
        \item ассоциативность $(A\cdot B)\cdot C = A\cdot (B\cdot C) $
        \item дистрибутивность1  $A(B+C) = AB + AC$
        \item дистрибутивноcть2 $\alpha (AB) = (\alpha A)B = A(\alpha B)$
    \end{itemize}
    \begin{definition}
        
    $\sphericalangle N\neq \mathbb{O}:\quad N^k = N\cdot N\cdot \ldots\cdot N = \mathbb{O}$

    N -- нильпотентная матрица, $k$ -- её порядок нильпотентности
    \end{definition}
    \begin{example}
        $\begin{bmatrix} 0& 1\\0&0\end{bmatrix} $
    \end{example}

    \begin{definition}
        Идемпотентной матрица называется, если $N^k = I$

        $k$ -- порядок идемпотентности

        
    \end{definition}

    \begin{example}
        $\begin{bmatrix} 1&0\\0&-1 \end{bmatrix} $
    \end{example}

    \begin{enumerate}
        \item [4] Транспонирование

            $A_{m\times n} = \| a_{ij} \|   $ Пусть $B = A^T = \| b_{ij} \| \implies  b_{ij} = a_{ji}$
    \end{enumerate}

    Свойства:
    \begin{itemize}
        \item $\left( \alpha \cdot A \right)^T = \alpha\cdot A^T $
        \item $(A+B)^T = A^T + B^T$
        \item $(AB)^T = B^T\cdot A^T$ -- проверить для себя
    \end{itemize}

    \begin{note}
        $A:\quad A=A^T$ -- симметричная/симметрическая матрица.Любая квадратная матрица с симметричными относительно диагонали элементами.

        $A:\quad A = -A^T$ -- антисимметричная матрица. На главной диагонали стоят нули

        $\begin{bmatrix} a_1&a_2&a_3\\ 0&a_4&a_5\\0&0&a_6 \end{bmatrix} $ -- верхняя треугольная. Транспонированная -- нижняя треугольная матрица
    \end{note}
\end{document}
