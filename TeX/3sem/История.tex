\documentclass{book}
%nerd stuff here
\pdfminorversion=7
\pdfsuppresswarningpagegroup=1
% Languages support
\usepackage[utf8]{inputenc}
\usepackage[T2A]{fontenc}
\usepackage[english,russian]{babel}
% Some fancy symbols
\usepackage{textcomp}
\usepackage{stmaryrd}
% Math packages
\usepackage{amsmath, amssymb, amsthm, amsfonts, mathrsfs, dsfont, mathtools}
\usepackage{cancel}
% Bold math
\usepackage{bm}
% Resizing
%\usepackage[left=2cm,right=2cm,top=2cm,bottom=2cm]{geometry}
% Optional font for not math-based subjects
%\usepackage{cmbright}

\author{Коченюк Анатолий}
\title{Наука и технологии в истории цивилизации}

\usepackage{url}
% Fancier tables and lists
\usepackage{booktabs}
\usepackage{enumitem}
% Don't indent paragraphs, leave some space between them
\usepackage{parskip}
% Hide page number when page is empty
\usepackage{emptypage}
\usepackage{subcaption}
\usepackage{multicol}
\usepackage{xcolor}
% Some shortcuts
\newcommand\N{\ensuremath{\mathbb{N}}}
\newcommand\R{\ensuremath{\mathbb{R}}}
\newcommand\Z{\ensuremath{\mathbb{Z}}}
\renewcommand\O{\ensuremath{\emptyset}}
\newcommand\Q{\ensuremath{\mathbb{Q}}}
\renewcommand\C{\ensuremath{\mathbb{C}}}
\newcommand{\p}[1]{#1^{\prime}}
\newcommand{\pp}[1]{#1^{\prime\prime}}
% Easily typeset systems of equations (French package) [like cases, but it aligns everything]
\usepackage{systeme}
\usepackage{lipsum}
% limits are put below (optional for int)
\let\svlim\lim\def\lim{\svlim\limits}
\let\svsum\sum\def\sum{\svsum\limits}
%\let\svlim\int\def\int{\svlim\limits}
% Command for short corrections
% Usage: 1+1=\correct{3}{2}
\definecolor{correct}{HTML}{009900}
\newcommand\correct[2]{\ensuremath{\:}{\color{red}{#1}}\ensuremath{\to }{\color{correct}{#2}}\ensuremath{\:}}
\newcommand\green[1]{{\color{correct}{#1}}}
% Hide parts
\newcommand\hide[1]{}
% si unitx
\usepackage{siunitx}
\sisetup{locale = FR}
% Environments
% For box around Definition, Theorem, \ldots
\usepackage{mdframed}
\mdfsetup{skipabove=1em,skipbelow=0em}
\theoremstyle{definition}
\newmdtheoremenv[nobreak=true]{definition}{Определение}
\newmdtheoremenv[nobreak=true]{theorem}{Теорема}
\newmdtheoremenv[nobreak=true]{lemma}{Лемма}
\newmdtheoremenv[nobreak=true]{problem}{Задача}
\newmdtheoremenv[nobreak=true]{property}{Свойство}
\newmdtheoremenv[nobreak=true]{statement}{Утверждение}
\newmdtheoremenv[nobreak=true]{corollary}{Следствие}
\newtheorem*{note}{Замечание}
\newtheorem*{example}{Пример}
\renewcommand\qedsymbol{$\blacksquare$}
% Fix some spacing
% http://tex.stackexchange.com/questions/22119/how-can-i-change-the-spacing-before-theorems-with-amsthm
\makeatletter
\def\thm@space@setup{%
  \thm@preskip=\parskip \thm@postskip=0pt
}
\usepackage{xifthen}
\def\testdateparts#1{\dateparts#1\relax}
\def\dateparts#1 #2 #3 #4 #5\relax{
    \marginpar{\small\textsf{\mbox{#1 #2 #3 #5}}}
}

\def\@lecture{}%
\newcommand{\lecture}[3]{
    \ifthenelse{\isempty{#3}}{%
        \def\@lecture{Lecture #1}%
    }{%
        \def\@lecture{Lecture #1: #3}%
    }%
    \subsection*{\@lecture}
    \marginpar{\small\textsf{\mbox{#2}}}
}
% Todonotes and inline notes in fancy boxes
\usepackage{todonotes}
\usepackage{tcolorbox}

% Make boxes breakable
\tcbuselibrary{breakable}
\newenvironment{correction}{\begin{tcolorbox}[
    arc=0mm,
    colback=white,
    colframe=green!60!black,
    title=Correction,
    fonttitle=\sffamily,
    breakable
]}{\end{tcolorbox}}
% These are the fancy headers
\usepackage{fancyhdr}
\pagestyle{fancy}

% LE: left even
% RO: right odd
% CE, CO: center even, center odd
% My name for when I print my lecture notes to use for an open book exam.
% \fancyhead[LE,RO]{Gilles Castel}

\fancyhead[RO,LE]{\@lecture} % Right odd,  Left even
\fancyhead[RE,LO]{}          % Right even, Left odd

\fancyfoot[RO,LE]{\thepage}  % Right odd,something additional 1  Left even
\fancyfoot[RE,LO]{}          % Right even, Left odd
\fancyfoot[C]{\leftmark}     % Center

\usepackage{import}
\usepackage{xifthen}
\usepackage{pdfpages}
\usepackage{transparent}
\newcommand{\incfig}[1]{%
    \def\svgwidth{\columnwidth}
    \import{./figures/}{#1.pdf_tex}
}
\usepackage{tikz}
\begin{document}
    \maketitle
    
    \chapter{Наука и технологии в истории цивилиации}

    Для 5 нужно в проектах участвовать.

    Рубежный и Итоговый тесты. 

    Ещё тесты в цдо, к котоорым готовиться по методичке там же

    Проект: в течение сеестра готовим работу, посвящённую аспекту научно-технического наследия Петра I. Предполагает научную литературу.

    исследовательская яработа и выстепление по ней

    инф. технологии


    Треки:
    \begin{itemize}
        \item Инженерное дело. Теория и правктика. Развитие инженерного дела: армия, флот, фонтанные сооружения, инж. уч. зав., огненные потехи, артиллерия, 
        \item нам Пётр I вёл науки.создание академии наук, пересаживание науки из западной Европы в Россию, кунсткамера, музеи, освоение теорриторий, географические экспедиции: Беринг, 
    \item Историческая память. Сохранение памяти, образа в скульптуре, топонимике и т.п. Здесь тоже форма: текст или видео или ещё что. устно всё равно нужно представить.
    \item Петровский парадиз и новая культура. Создание Снкт-Петербурга, регулярная планировка, уникальность, культура СПб и новая культура России, которая возникаа здесь, балы, театры, фейерверки
    \end{itemize}

    \section{Наука и техника в истории первобытного мира на древнем востоке.}

    Учебник: История науки и техники, учебно-метолическое пособие 

    Темы: мегалиты на территории Северной Европы, Тайны Египетских пирамид, Астрономия в древнем Вавилоне, математика древней Индии: десятичная запись, отличие от геометрической Греции, ввод Алгебры, медицина древнего Китая

    \subsection{Счёт времени и появление письменности}

    Как люди считали время? 


    \begin{enumerate}
        \item 
    первое, самое простое -- сутки, смена дня и ночи. Первобытный человек мог это заифксировать и вести отсёт 
\item Лунный цикл (29,5 суток)
\item Времена года и год (лунные месяца поначалу) 
\item Египтяне разработали первый солнечный календарь, 365 суток. Разливался Нил и сезона было 3 по разливам Нила. Очередной год -- Сириус восходил вместе с солнцем, место восхода солнца начали отслеживать.
\item Юлианский календарь (46 до н.э.) Метонов цикл, вставные месяцы, солнечный и лунный цикл привести в соответствие, Астроном Созиген прибыл в Рим и разработал более точный календарь, который учитывал 365/25? 4-х летний цикл
\item Григорианский календарь. Случился сдвиг на 13 дней за полторы тысячи лет. Там, пасха стала нетогда и вручную откорректировали более точно с 1582 г. Пришёл в Россию только в 1918. Появилась разница между нашим и католическим рождеством, потому что церковь осталась на Юлианском. Старый новый год тоже с этим связан.
    \end{enumerate}

    первобытные люд отсчитывали годы от какого-то значимого события для коллектива.

    \subsection{Хронология}

    Эра -- точка отсчёта. Первые системы их не имели
    \begin{enumerate}
        \item Древневосточные: цикличные как 12-летние в Китае, либо по правителям или династиям как в Египте.
        \item По выборным должностным лицам. Но они выбирались каждый год и года превращались в списки имён
        \item Датировка по олимпиадам (с 776 г до н.э.) Номер олимпиады и 1,2,3 год после неё
        \item Древняя Иудея. Связана с верхним заветом. Сотворение мира -- точка отсчёта. 5508 г до н.э.
        \item Рим -- 753 г. до н.э. Аb Urbe Condita AUC
        \item 525 г  -- Дионисий расчитал дату рождения Христа по некоторым астрономическим явлениям в Библии. Нулевого года нет. За 1 до н.э. идёт 1 н.э. Чуть позже Астрономы посчитали, что Дионисий ошибся и Иисус родился в 4 г до н.э.
    \end{enumerate}

    \section{Введение в историю науки и техники}
    \subsection{Что такое наука?}

    \begin{definition}
        Наука -- любое познание, которое ведёт человек, но это слишком широкое определение. Человек начинает познавать мир, как только появляется.

        Наука -- проявления действия в человеческом обществе совокупной человеческой мысли (В.И. Вернандский)

        Наука -- есть познание с рефлексией (изучает в том числе и само себя) и доказательством. (В.Е. Еремеев) Не раньше древней Греции -- первые научные парадигмы с доказательствами

        Наука -- сфера человеческой деятельности, функцией которой является выработка и теоретическая систематизация объективных знаний о действительности (Философский энциклопедический словарь)

        Наука -- способ удовлетворить своё любопытство за государственные деньги
    \end{definition}

    Происхождение науки:
    \begin{itemize}
        \item По Вернандскому наука произошла из религии. ``В религиозных сознаниях отливались добытые в практической деятельности людей знания, который за счёт этого входили в сознания людей'
        \item Проиходит из магии
    \end{itemize}

    Обе исходят из следующих принципов:
    \begin{enumerate}
        \item Одно событие следует за другим
        \item Порядок и единообразие природных явлений
        \item Стремление к установлению повторяющейся последовательности событий 
    \end{enumerate}

    \subsection{Проблема возникновения науки}
    Три точки зрения:
    \begin{itemize}
        \item Наука как любая познавательная деятельность человека рождается вместе с человеком -- каменный век. Сложный быт $\to $ арифметика, Агрокультура -- исследования в эту сторону
        \item Наука как первая программа исследования природы. Древняя Греция, Индия, Китай. VI-V вв до н.э. В Индии и Китае связано с духовными течениями и появлением научных парадигм. Осевое время ~800 лет, появление философии и различных религий
        \item Наука в современном смысле. Экспериментальное подтверждение
    \end{itemize}

    \section{Что такое техника}

    \begin{definition}
        Техника -- обобщающее наименования сложных устройств, механизмов, систем, а также методов, процессов и технологий упорядоченной искусной деятельности.
    \end{definition}

    Техника свойственна не только человеку, но и животным. Пауки плетут паутины, грачи используют водоизмещение для добывания пищи..

    Наука и Технология очень долгое время были не связаны.

    методы научного познания:
    \begin{itemize}
        \item Индуктиыный -- от частного к общему, т.е. от единичных фактов в обощению
        \item Дедуктивный -- от общего к частному -- выдвижение гипотезы и затем её проверка эмпирическими данными.
    \end{itemize}

    Международное научное сообщество и объём научной информации удваиваются раз в 10 лет. Это считается одной из причин замедление развития теоретической науки.

    Цель курса -- кроме научных результатов описать социокультурных и мировоззренческий контекстом творчества учёных и факторов, тормозивших развитие научных идей.

    Модели развития науки:
    \begin{enumerate}
        \item кумулятивисткая. 3 стадии по О. Конту:
            \begin{enumerate}
                \item Теологическая (причины явления -- сверхъестественные)
                \item Метафизическая (причины явления -- абстрактные сущности, вода, воздух, земля)
                \item Позитивная (привила явления -- неизменные законы природы). Наука -- критерий истинности.
            \end{enumerate}
            Научные знания накапливаются и постоянно идёт прогресс. Линейная модель, но наука не шла ровно, поэтому другие модели:
        \item революционная: Алексадр Койре

            Начная революция -- такой вид новаций в науке, который кардинально меняет основные научные традиции. 3 вида научных революций:
            \begin{enumerate}
                \item Возникновение новых фундаментальных теорий
                \item Внедрение новых методов исследования
                \item Открытие новых ``миров'' (атомов и молекул, галактик, кристаллов, вирусов и др.)
            \end{enumerate}
        \item Ситуационная

            В 1970 приобретает значимость понятие ``ситуационные исследования'' Case Studies. В таких исследованиях ставится задача понять прошлое событие не как вписывающееся в единый ряд развитии, не как обладающее какими-то общими с другим событиями чертами, а как своеобразное, невозпроизводимое в других условиях.
    \end{enumerate}

    Факт и источник в истории:
    \begin{itemize}
        \item Исторический факт -- действительное, невымышленное происшествие, событие, явление в истории, которое может быть использовано для какого-либо заключения, вывода и является проверкой для предположения.
        \item Исторический источник -- письменный памятник или материальный памятник, на основе которого строится историческое исследования.
    \end{itemize}

    наличие источников, сносок с ними -- то, что определяет книги по историю написанную учёным.

    Источник, особенно письменный, -- всегда авторское произведение. Чтобы выделить в нём факты учёные используют инструмент критики.

    \begin{example}
        Доказательство подлинности Слова о полку Игореве. Существовала в единственном экземляре. Она сгорела в Москве. Была единственная копия, которая считалась подделкой. Но в берястеных грамотах, когда их обнаружили нашли обороты речи, которые были использованы в слове о Полку Игореве. Чтобы их использовать автор должен был бы знать об этих грамотах, которые откопали через полтора века.
    \end{example}

    Периодизация человеческой истории:
    \begin{itemize}
        \item Появление человка -- 2 млн. назад
        \item Ранний Паеолит -- 2 млн. назад -- 200 тыс назад
        \item Средний Палерлит -- 200 тыс -- 40 тыс
        \item Возникновение первобытного общества
            \item Нижний Палеолит -- 40 тыс -- 12 ты назад
            \item Мезолит -- 12 тыс --  6 тыс назад
            \item Неолит -- 6 тыс -- 4 тыс -- производство продуктов питания, земледелие и скодовоство
            \item Появление государственности, с которой принято связывать понятие цивилизации -- около 6 тыс назад
    \end{itemize}

    \subsection{Доцивиализационный период}

    эолиты -- затачивали один камень другими, чтобы получить инструмент. В первую очередь делались из кремня, а также яшма, роговик, халцедон, гранитный валун

    Прошло  много сотен тысяч лет между поддерживанием огня и его искусственном добывании. (высечении, выскабливании, выпиливании, высверливании)

    В позднем палеолите появляются составные орудия, каменные топоры с деревянной рукояткой, усиливающе в 2-3 раза сиул и эффективность орудия. Позднее появляются лук и стрелы.

    \subsection{Первый глобальный продовольственные кризис}

    Относительное перенаселение планеты. Чтобы прокормит одного человека охотой и собирательством необходимо 2 км${}^2$ площади. При эффективном земледелии достаточно $100m^2$

    \subsection{Неолитическая революция}

    ``Неолитическая революция'' (термин Гордона Вир Чайлда) -- переход от присваивающих к производящим формам хозяйствам. Появление технологии регулируемого обжига глины, шлифовальных каменных орудий, ткачества, сельскохозяйственных орудий (мотыга, серп). Стало возможно хранить еду в глиняных изделиях

    Развитие металлургии:
    \begin{itemize}
        \item Схема Р. Форбеса
            \begin{enumerate}
                \item Самородный металл как камень
                \item Самородный метал, обработанный ковкой (золото, серебро, метеоритное железо)
                \item Рудная металлургия (из руд получали медь, свинец, серебро, сурьму, сплавы меди)
                \item Металлургия железа.
            \end{enumerate}
        \item Схема Г. Коглена
            \begin{enumerate}
                \item Холодная, а в дальнейшем горячая ковка меди
                \item Плавление самородной меди в открытые формы простых изделий
                \item <..>
            \end{enumerate}
    \end{itemize}

    \subsection{Бронзовый век 2-3 тыс лет до н.э.}

    Металлургия бронзы Появление кочевого скотоводства и поливного земледелия. Появление письменности и первых цивилизаций. С появлением государственности развивается техника, водоснабжение, орошение, методы производства.

    В Китае не было металлургии меди.

    С 1 тыс до н.э. Сразу начался железный век
    \section{Практика}

    Палеоконтакт: тектиты, тунгусский метеорит, племя догонов

    научное наблюдение -- главное, чтобы наблюдатель не вмешивался в процесс.

   пирамиды, остров Пасхи, Баольбек

   \section{Наука и техника дальневосточных цивилизаций (Египта, Месопотамии, Индии и Китая)}

   Процветание этих стран -- результат ирригационногоземледелия
   \subsection{Первая промышленная революция -- 3 тыс. лет до н.э.}

   \begin{itemize}
       \item Изоберетение колеса
       \item обжиг кирпича
       \item гончарный круг -- керамика -- сосуды
       \item изобретение плуга
       \item изобретение вёсельного и парусного судов
       \item изобретение весов, отвеса, уровня, угломера, циркуля -- более точное строительство, зачатки геометрических знаний
       \item изобретение кузнечных мехов
       \item рычаг, клин, домкрат, сифон, водяные часы
   \end{itemize}

   \subsection{Египет}

   Периоды:
   \begin{itemize}
       \item 3000-2000 лет до н.э. -- древнее царство, иероглифы, пирамиды
       \item 2000-1800 лет до н.э. -- среднее царство, развитие математики, звёздные календари
       \item 1600-1100 лет до н.э. -- новое царство, создание Египетской империи
       \item 1100-30 лет  до н.э. -- позднее царство, Египет под властью иноземных династий
   \end{itemize}

   Письменность -- одна из древнейших в мире. пиктографические знаки.

   3 вида письменности:
   \begin{itemize}
       \item Иероглифическая
       \item Иератическая -- священное письмо
       \item Демотическая -- курсив, которым писали простые люди
   \end{itemize}

   1799 -- находка РОзеттского камня. одини тот же текст на иероглифике, демотике и древнегреческом.

   1822 -- дешифровка египетских иероглифов Франсуа Шампольоном. Использовалось знание копского языка, который считается приемником.

   Находка Гаса Юнкера -- каменные шары диаметром от 12 до 40 сантиметров сделанные из долерита --камня очень твёрдой породы. Юнкер предположил, что это были шарикоподшипники. А 1936 он провёл опыт, который показал, что всего один человек может с помощью таких шариков может передвигать блок весом в несколько тонн.

   Пирамиды -- около 2650 до н.э

   Медицина Египта известна нам по двум папирусам:
   \begin{itemize}
       \item Эдвина Смита -- точное описание органов человеческого тела. Мозг понимается как центр мышления.
       \item Эбер -- более 900 рецептов
   \end{itemize}

   Обряд мумификации

   Математика: десятичная, но не позиционная. Знали только натуральные дроби. Умножение -- удвоение и сложение. Задачи на нахождение суммы геометрической прогрессии. Вычисление со вспомогательными числами.

   Папирус Ринда -- задача о 7 кошках.

   Техника:
   \begin{itemize}
       \item Бурение медной трубки с помощью кварцевых зёрен
       \item 7 тыс. лет назад египтян применяли простейший сверлильный станок с лучковым приводом.
   \end{itemize}

   \subsection{Месопотамия}

   Обзор: междуречье Тегра и Ефрата
   \begin{itemize}
       \item 3000 лет до н.э. -- возникновение шумерских городов-государств
       \item 2400-2100 -- государство Аккад объединяет Месопотамию
       \item 2000-1600 -- возвышение Вавилонского царства
       \item 1600-1100 -- средневавилонский период
       \item 1100 - 539 -- нововавилонский период. Возвышение и упадок Ассирии.
   \end{itemize}

   Письменность:
   \begin{itemize}
       \item  Ключ к расшифровке клинописи был найден немцем Георгом Фридрихом Гротефендом в 1902 г.
       \item Датчанин Эдвард Хинкс доказал, что вавилонская клинопись является слоговым, а не алфавитным письмом.
   \end{itemize}

   Шумерская и Вавилонская математика: первая позиционная система счисления. Но 60-ричная. Отсюда деление часа, угла на 60.

   Таблицы умножения и таблица квадратных корней. Суммирование арифметических прогрессий. Решение систем уравнений с двумя неизвестными. Вычисление площади треугольника и трапеции, объёма цилиндра и призмы.

   Первая библиотека Ашшурбанипала -- древнейшая из всех известных библиотек. 25 000 - 30 000 глиняных табличек с клинописными текстами. Библиотека горела, но таблички сделались только более прочными

   Астрономия -- лунно-солнечный календарь, 12-13 месяцв, вычисление лунных затмений, таблицы положений отдельных звёзд. Греки опирались на вавилонские налюдения.

   Астрология предзнаменований (Энума Ану Энлиль) и зодиакальная астрология.

   Вильгельм Кёниг в раскопках близ Багдада обнаружил предмет, похожий на гальваническую батарею.

   Глиняный сосуд, в котором был железный стержень и запаянный медный цилиндр. Изучи предмет выяснили, что электролит представлял собой уксусный кислоты, а обмазка -- битум.

   Возможное применение: гальваностегия, то есть покрытие золотом серебро при помощи электролиза. Было проверено эксперементально, что даже с таким слабым током это возможно. Но едиснтва среди учёных нет.

   \subsection{Индия}

   Периоды:
   \begin{itemize}
       \item  3000-2000 -- Хараппская цивилизация в долине Инда
       \item 2000-1400 -- вторжение ариев, кастовая система
       \item До VI вв до н.э. -- ведийский период. древнеиндийская религия главная
       \item V вв до н.э.  -- водникновение буддизма. Независимые княжества.
       \item III -- появлении империи Маурьев
       \item II -- распад империи
   \end{itemize}

   Математика:
   \begin{itemize}
       \item  Изобретение десятичной позиционной системы счисления, приянтой у нас (около 500 г н.э.)
       \item создание развитой алгебраической символики
       \item Решение неопределённых уравнений при определении периодов повторения одинаковых относительных положений небесных тел с различным периодом обращения.
   \end{itemize}щ
   Астрономия -- Сурья-сиддханта -- астрономическое сочинение. Основано на греческой теории эпициклов. эпциикл -- круг, центр которого движется по большому кругу вокруг Земли. Планета, описывая эпицикл, имеет одновременно два круговых движения.

   \subsection{Китай}

   Периоды:
   \begin{itemize}
       \item ок 1600-1000 -- государство Шан, совоение бронзы, появление письменности
       \item 1000-221 -- государство Сжоу. С V вв до н.э. освоение железа
       \item 221до н.э. -220 н.э -- Империи Цинь и Хань. Строительство Великой китайской стены.
       \item IV-VI -- нашествие кочевников, распад Империи
   \end{itemize}

   Особенности китайской науки:
   \begin{itemize}
       \item  Китвйская наука обладала представлениями, бизкими к теории поля, что способствовало изобретению компаса
       \item Китайская физика тяготела к волновым теориям и не признавала атомистику
   \end{itemize}

   Вопросы Нидэма:
   \begin{itemize}
       \item  Почему в течении 2000 лет до начала научной революции XVII китайская наука опережала западную
       \item Почему современная наука возникла в Европе, а не Китае
   \end{itemize}

   Академия Цзи-ся -- существовала в IV-III вв до н.э. В ксловиях официально поощряемой дискуссии в Цзися уточнялись теоретические позиции и рождались первые формы синтеза основных идейных направлений китайской философии. В итоге сложилось и собственно учение Цзися, отмечанное общей даосской ориентацей.
   
    Математика и атсрономия:
    \begin{itemize}
        \item Использование отрицательные числа и десятичные дроби. Использовалось десятичная и двоичная системы счисления. 
        \item Концепция куполообразного неба -- неба полусфера в 8000 лин от земли
        \item Вычисление продолжительности года в 365.25
    \end{itemize}

    Китайская техника:
    \begin{itemize}
        \item Изоберетение компаса -- IV до н.э.
        \item Изобретение сейсмографа в 31 г н.э. -- большой медный сосуд с медным цилиндром-маятником. При толчке маятник жал на рычажок, который открывал пасть драгона и медный шарик выпадал в рот лгушке
        \item Изобретение бумаги в 105 г н.э.
    \end{itemize}


    \section{Наука и техника древней Греции и Рима}

    Периодизация Греции:
    \begin{itemize}
        \item 1100-800 -- гомеровский период (``тёмные века'')
        \item 800-500 -- архаический период
        \item 500-336 -- классический период
        \item 336-30 -- эллинистический период
    \end{itemize}

    Периодизация Рима:
    \begin{itemize}
        \item  750-510 -- царский период
        \item 510-30 -- республиканский период
        \item 30 до н.э - 476 н.э. -- императорский период.
    \end{itemize}


    \subsection{Особенности античной науки и техники}

    Греческое чудо -- почти внезапное возникновение стремительного развития в Древней Греции начиная с VI вв. до н.э философии и науки, которые дали могучий толчок всему дальнейшему интеллектуальному прогрессу. Почему именно там, именно так внезапно?

    агональное начало греческой цивилизации
    \begin{itemize}
        \item  Агон -- борьба или состязание. Греческая культура была проникнута состязательным началом, которое дало науке и философии диалог как метод познания
        \item ``Будь лучшим. Даже боги соревнуются.'' Обретением бессмертия греки считали сохранение их в памяти поколений.
    \end{itemize}

    Предпосылки ``греческого чуда''
    \begin{itemize}
        \item  распространение железа и увеличение производительности труда
        \item Великая греческая колонизация -- контакты с другими народами 
        \item Развитие полисной демократии и расширение личных свобод граждан
    \end{itemize}

    Особенности
    \begin{itemize}
        \item  Пренебрежительное отношение к физическому труду и практической деятельности (то, что делают рабы, а не свободный ты)
            \begin{itemize}
                \item Акцент на идеальном, а не на метериальном
                \item Преобладание теоретических методов познания
            \end{itemize}
        \item Недостаток сырьевых и энергетических ресурсов

            Трудности в реализации смелых технических идей
        \item Античное хозяйство не было нацелено на расширенное воспроизводство из-за узости внутреннего рынка

            Отсутствие необходимости в рационализации труда и применении машин
    \end{itemize}
    \subsection{Возникновение античной науки из философиии в VI-IV до н.э.}

    Иония -- контактный решион, где греческая цивилизация встречалась с древневосточными

    В VI вв до н.э. здесь возникает первая философская школа -- натурфилософов

    Фалес Милетский: Что является основой, началом мира
    \begin{itemize}
        \item  Вода -- первосубстанция, а испарение и конденсация -- универсальные процессы мироздания
        \item Обосновал ряд теорем в геометрии (например, что вписанный в полукруг угол, построенный на диаметре, всегда будет прямым)
        \item Определили дни солнцестояний и равноденствий, ввел календарь из 12 тридцатидневных месяцев. Предположил, что Луна светит не своим светом.
    \end{itemize}

    Фалес Милетский (640/624 -- 548/545 до н.э)

    \begin{itemize}
        \item Считал воду первосубстацией, а испарение и конденсацию -- универсальными процессами
        \item Обосновал ряд теорем в геометрии
        \item Определил дни солнцестояний и равноденствий, ввёл календарь из 12 тридцатидневных месяцев. Предположил, что Луна светит не своим светом.
    \end{itemize}

    Анаксимандр (611-546 до н.э.)

    \begin{itemize}
        \item Считал первосубстанцией апейрон (неопределённое), из которого выделились противоположности: влажного и сухого, хлодного и теплого.
        \item Земля -- плоский цилиндр высотой в $\frac{1}{3}$ вокруг которого вращаются звёзды, Луна и Солнце
        \item Считал, что жизнь зародилась на границе моря и суши
    \end{itemize}

    Приписывается создание Гномона -- простейших солнечных часов

    Анаксимен (585-525 до н.э.)
    \begin{itemize}
        \item  В качестве первоначала признавал воздух, из которого за счёт разрежения и сгущения происходят остальные стихии
        \item Полагал, что землетрясения происходят из-за того, что земля, увлажняясь и высыхая, покрывается трещинами, вследствие чего ``расколотые таким образом возвышенности рушатся вниз''
        \item Первый правильно расположил светила
    \end{itemize}


    Пифагор с острова Самос (580-500 до н.э)
    \begin{itemize}
        \item Числа отражают структуру Космоса, задают форму вещей
        \item Пифагор пришёл к выводу, что Земля представляет собой шар.
        \item Доказал теорему Пифагора
        \item Первый пришёл к выводу, что Земля -- шар
    \end{itemize}
    Монохорд -- инструмент, служащий для точного построения интервалов путём фиксации различных длин звучащей чащи

    Пифагорийцы:
    \begin{itemize}
        \item Гиппас из Метапонта -- открыл иррациональные числа
        \item Алкмеон -- вскрытия животных и утверждение о том, что мозг -- центр мышления
        \item Архит -- исследование арифметической и геометрической пропорций
    \end{itemize}

    Учение о переселении душ -- психос -- после смерти по тому как человек жил, его душа переселяется в другое существо

    Атомисты

    Демокрит из Абдер (460 -- 370 до н.э.)
    \begin{itemize}
        \item  Существуют только атомы и пустота. Атомос -- неделимый. Атомы -- это неизменные части бытия
        \item Атом -- неуничтожимая, неделимая, плотная, непроницаемая частица
        \item Между атомами всегда пустота
        \item Атомам присуще движения
        \item Различные движения и конфигурации атомов и скоплений атомов -- это причины всех явлений природы.
    \end{itemize}

    Е.Л. Фейнберг о Демокрите и Эйнштейне
    \begin{itemize}
        \item  Демокрит, располагая лишь восковой дощечкой и своим разумом смог предположить, что материя состоит из атомов за века до того, как человечество сможет проникнуть внутрь вещества
        \item Эейнштейн, располагая лишь бумагой, карандашом и собственным разумом, смог предположить, что ..
    \end{itemize}

    Софисты и Сократ: поворот в Философии

    ``Я знаю только то, что ничего не знаю, но другие не знают и этого'' -- Сократ

    ``Человек -- есть мера всех вещей: существующих, что они существуют, а несуществующих, что они не существуют''

    Платон (427 -- 347)
    \begin{itemize}
        \item  Ученик Сократа
        \item В 387 основал в Афинах философскую школу, которая называлась Академией
        \item По Платону существует два мира -- мир вещей и мир идей. Предметы материального мир -- это тени, отражения идей. Истинный мир -- сир идей. В центре мира идей -- высшая идея блага.
    \end{itemize}

    Аристотель (384 -- 322)
    \begin{itemize}
        \item Является основателем логики, психологии и биологии
        \item Материя не обладает конечной делимостью, она континуальна. В природе нет пустоты
        \item Вселенная не бесконечна и ограничена сферой неподвижных звёзд
        \item В основе природы лежат не числа, а форма и материя, определяющие в своём соединении качества вещей.
        \item Считал, что для постоянного движения нужна постоянная сила. Аристотелевская физика. Снаряд летит, потому что его толкает воздух.
    \end{itemize}

    Первые астрономические теории греков:
    \begin{itemize}
        \item Филолай -- ``теория гестиоцентризма'' (в центре Вселенной ``Огненный очаг'', Гестия)
        \item Евдокс Книдский -- теория гомоцентрических сфер для объяснения движения планет.
        \item Гераклит Понтийский -- теория геогелиоцентризма (планеты вращаются вокруг солнца, а солнце вокруг Земли)
    \end{itemize}

    Возникновение медицины:
    \begin{itemize}
        \item  В древней Греции различалось храмовое врачевание и эмпирическое.  VI до н.э. пявляются первые святилища в честь Асклепия -- асклепионы (всего их было около 300 к конце IV до н.э.). Самый величественный располагалась в Эпидавре
            Создателем теоретической медицины стал Гиппократ, который обосновал теорию о четырёх телесных соках -- кровь, флегма, желчь и чёрная желчь
            Причина болезни -- дисгармония этих жидкостей. Суть лечения -- восстановить баланс. На человеческий иммунитет не нужно никак воздействовать, нужно лишь дать хорошие условия для самовосстоновления.
    \end{itemize}
    
    Самосский туннель -- строительная технология:
    \begin{itemize}
        \item  Около 530 до н.э. по повелению могущественного тирана Поликрата Евпалин построил водопровод через известняковую гору Кастро на острове Самос
        \item Туннель Евпалина был почти прямолинейным. Это стало возможным благодаря диоптру, горизонтальной линейки с двумя смотровыми отверстиями, которую можно поворачивать и при помощи которой на плоскости  можно визировать прямые углы.
    \end{itemize}
    \subsection{Расцвет античной науки в эллинистический период}

    \begin{itemize}
        \item  Александрийский Мусейдон -- основан в начале III века до н.э. при Птолемее Сотере находился на государственном обеспечении. В состав мусейдона входила обширная Александрийская библиотека, организованная в этот же период.
        \item В Месейдоне обучались или работали крупнейшие учёные эпохи эллинизма.
    \end{itemize}

    Математика:
    \begin{itemize}
        \item  Архит из Тарента прославился прежде всего решением задачи на удвоение куба ``методом сведения'' ее к задаче нахождения двух средних пропорциональных между двумя отрезками, из которых один в два раза больше другого.
    \end{itemize}

    Астрономия:
    \begin{itemize}
        \item Аристарх Самосский (конец IV -- начало III до н.э.)-- предположил космологическую модель, в которой Земля и другие планеты движутся вокруг Солнца -- первая в мире гелиоцентрическая система
        \item Аполлоний Пергский (262-190 до н.э.) предложил теорию эпициклов. Согласно этой теории, планета движутся по малому кругу вокруг центра, который движется по большому кругу вокруг Земли.д
    \end{itemize}

    География
    \begin{itemize}
        \item  Эратосфен -- применил к поверхности Земли систему координат, вычислив длину земной окружности в 40 225 км, что на 225 км отличается от истинной величины. Этот результат оставался непревзойённвм до XVII в.
    \end{itemize}

    Медицина
    \begin{itemize}
        \item  В Александрии анатомирование тел умерших и живосечения производились Герофилом их Халкедона (385 -- 280 до н.э.), Эраситратом (300 -- 240 гг до н.э.). Описали артерии и вены, ...
    \end{itemize}

    \subsection{Наука древнего Рима и упадок античной науки}

    Технические достижения:
    \begin{itemize}
        \item  Строительство мощёных дорог на территории Европы
        \item Строительство акведуков (общая протяжённость 440км)
        \item Создание системы центрального отопления для общественных бань (гипокауст)
    \end{itemize}

    Плиний Старший (23-79)
    \begin{itemize}
        \item  Автор ``Естественной истории'' в 37 книгах -- энциклопедия по известным областям знаний
        \item ``мы не можем не только добавить каких-нибудь новых исследований, но и основательно усвоить старые.''
    \end{itemize}

    Медицина в Риме:
    \begin{itemize}
        \item Развитие военной медицины и хирургии
        \item Авл Корнелий Цельс -- составил медицинский справочник популярный в средние века.
        \item Гален их Пергама (II н.э) -- написал много трудов по медицине, изучал анатомии животных, опроверг распространённое мнение о наполненности артерий воздухом.
    \end{itemize}

    Астрономия в Риме
    \begin{itemize}
        \item Клавдий Птолемей (100-170) -- последний великий астроном античности.
        \item Завершил построение геоцентрической системы мира, в которой движение Солнца, Луны и пяти планет описывались с помощью системы эксцентрических кругов и эпициклов.
    \end{itemize}

    Математика в Риме
    \begin{itemize}
        \item Диофант из Александрии (III в н.э.) написал ``Арифметику'' в 13 книгах, представлявших собой собрание задач, решаемых независимо друг от друга и в большинстве своём приводящихся к неопределённым уравнений до 4-й степени. Ввёл буквенную символику для обозначения алгебраических выражений, знаки для сложения и вычитания.
    \end{itemize}


    \subsection{Античная техника и её достижения}

    \begin{itemize}
        \item  При сравнении античной техники с техникой других цивилизаций древности, будь то Египет или Вавилон, Китай или Индия, оказывается,что технический арсенал античности превосходил все существование на тот периоды общества.
    \end{itemize}

    Военная техника
    \begin{itemize}
        \item Использовалось несколько формул для расчёта диаметра канал,в  котором находились упругие натянутые жилы, с помощью которых метали стрелы и ядра. $D = 1.1 \sqrt[3]{m\times 100} $
    \end{itemize}

    Водяной насос Ктесибия. использовался при пожаротушении.

    Архимед (287 -- 212)
    \begin{itemize}
        \item Архимед был автором водоподъемного винта. Занимался сложными математическими кривыми, в том числе и спиралями.
        \item Среди других практических изобретений, упомянем систему подъёмных блоков (полиспат), предназначенную для спуска на воду корабля и его подъёма на сушу.
    \end{itemize}

    Герон Александрийский: Ветряная мельница и паровой двигатель, которые получат распространение только через 10 и 17 веков после него.



\end{document}
