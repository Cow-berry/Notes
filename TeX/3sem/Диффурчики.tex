\documentclass{book}
%nerd stuff here
\pdfminorversion=7
\pdfsuppresswarningpagegroup=1
% Languages support
\usepackage[utf8]{inputenc}
\usepackage[T2A]{fontenc}
\usepackage[english,russian]{babel}
% Some fancy symbols
\usepackage{textcomp}
\usepackage{stmaryrd}
% Math packages
\usepackage{amsmath, amssymb, amsthm, amsfonts, mathrsfs, dsfont, mathtools}
\usepackage{cancel}
% Bold math
\usepackage{bm}
% Resizing
%\usepackage[left=2cm,right=2cm,top=2cm,bottom=2cm]{geometry}
% Optional font for not math-based subjects
%\usepackage{cmbright}

\author{Коченюк Анатолий}
\title{Дифференциальные уравнения}

\usepackage{url}
% Fancier tables and lists
\usepackage{booktabs}
\usepackage{enumitem}
% Don't indent paragraphs, leave some space between them
\usepackage{parskip}
% Hide page number when page is empty
\usepackage{emptypage}
\usepackage{subcaption}
\usepackage{multicol}
\usepackage{xcolor}
% Some shortcuts
\newcommand\N{\ensuremath{\mathbb{N}}}
\newcommand\R{\ensuremath{\mathbb{R}}}
\newcommand\Z{\ensuremath{\mathbb{Z}}}
\renewcommand\O{\ensuremath{\emptyset}}
\newcommand\Q{\ensuremath{\mathbb{Q}}}
\renewcommand\C{\ensuremath{\mathbb{C}}}
\newcommand{\p}[1]{#1^{\prime}}
\newcommand{\pp}[1]{#1^{\prime\prime}}
% Easily typeset systems of equations (French package) [like cases, but it aligns everything]
\usepackage{systeme}
\usepackage{lipsum}
% limits are put below (optional for int)
\let\svlim\lim\def\lim{\svlim\limits}
\let\svsum\sum\def\sum{\svsum\limits}
%\let\svlim\int\def\int{\svlim\limits}
% Command for short corrections
% Usage: 1+1=\correct{3}{2}
\definecolor{correct}{HTML}{009900}
\newcommand\correct[2]{\ensuremath{\:}{\color{red}{#1}}\ensuremath{\to }{\color{correct}{#2}}\ensuremath{\:}}
\newcommand\green[1]{{\color{correct}{#1}}}
% Hide parts
\newcommand\hide[1]{}
% si unitx
\usepackage{siunitx}
\sisetup{locale = FR}
% Environments
% For box around Definition, Theorem, \ldots
\usepackage{mdframed}
\mdfsetup{skipabove=1em,skipbelow=0em}
\theoremstyle{definition}
\newmdtheoremenv[nobreak=true]{definition}{Определение}
\newmdtheoremenv[nobreak=true]{theorem}{Теорема}
\newmdtheoremenv[nobreak=true]{lemma}{Лемма}
\newmdtheoremenv[nobreak=true]{problem}{Задача}
\newmdtheoremenv[nobreak=true]{property}{Свойство}
\newmdtheoremenv[nobreak=true]{statement}{Утверждение}
\newmdtheoremenv[nobreak=true]{corollary}{Следствие}
\newtheorem*{note}{Замечание}
\newtheorem*{example}{Пример}
\renewcommand\qedsymbol{$\blacksquare$}
% Fix some spacing
% http://tex.stackexchange.com/questions/22119/how-can-i-change-the-spacing-before-theorems-with-amsthm
\makeatletter
\def\thm@space@setup{%
  \thm@preskip=\parskip \thm@postskip=0pt
}
\usepackage{xifthen}
\def\testdateparts#1{\dateparts#1\relax}
\def\dateparts#1 #2 #3 #4 #5\relax{
    \marginpar{\small\textsf{\mbox{#1 #2 #3 #5}}}
}

\def\@lecture{}%
\newcommand{\lecture}[3]{
    \ifthenelse{\isempty{#3}}{%
        \def\@lecture{Lecture #1}%
    }{%
        \def\@lecture{Lecture #1: #3}%
    }%
    \subsection*{\@lecture}
    \marginpar{\small\textsf{\mbox{#2}}}
}
% Todonotes and inline notes in fancy boxes
\usepackage{todonotes}
\usepackage{tcolorbox}

% Make boxes breakable
\tcbuselibrary{breakable}
\newenvironment{correction}{\begin{tcolorbox}[
    arc=0mm,
    colback=white,
    colframe=green!60!black,
    title=Correction,
    fonttitle=\sffamily,
    breakable
]}{\end{tcolorbox}}
% These are the fancy headers
\usepackage{fancyhdr}
\pagestyle{fancy}

% LE: left even
% RO: right odd
% CE, CO: center even, center odd
% My name for when I print my lecture notes to use for an open book exam.
% \fancyhead[LE,RO]{Gilles Castel}

\fancyhead[RO,LE]{\@lecture} % Right odd,  Left even
\fancyhead[RE,LO]{}          % Right even, Left odd

\fancyfoot[RO,LE]{\thepage}  % Right odd,something additional 1  Left even
\fancyfoot[RE,LO]{}          % Right even, Left odd
\fancyfoot[C]{\leftmark}     % Center

\usepackage{import}
\usepackage{xifthen}
\usepackage{pdfpages}
\usepackage{transparent}
\newcommand{\incfig}[1]{%
    \def\svgwidth{\columnwidth}
    \import{./figures/}{#1.pdf_tex}
}
\usepackage{tikz}
\DeclareMathOperator{\Dom}{Dom}
\begin{document}
    \maketitle
    %
    %
    %
    % Сашин блок
    
    Связь по: mvbabushkin@itmo.ru --- просьба писать именно на почту
30 --- экзамен 
70 --- практика (будет уточняться)

Литературу пришлю

\section{Введение}
\subsection{Уравнения первого порядка}
Допустим, $y$ --- неизвестная величина. Заметим, что это не просто число, а некоторая зависимость (например, температура, зависящая от времени), то есть это некоторая искомая функция. Ну, и часто непосредственно, нам не написать чему она равна; явно эту функцию просто так не напишешь, но можно написать некую взаимосвязь между этой функцией и переменной, возможно еще и производной и т.п.

\begin{definition}
	Такая взаимосвязь называется \textbf{дифференциальным уравнением}.
\end{definition}

\begin{example}
Допустим, у нас есть кролики, заведем таблицу и будем считать кроликов каждый день.

Предположем, мы смотрели на эксперемент и обнаружили такую зависимость:
Прирост примерно пропорционален текущему клличеству и времени замерки.

\[y_{k + 1} - y_k \approx \alpha  y_k (t_{k + 1} - t_k)\]

Так же заметим, что если узмельчатьь шаг времени, то зависимость будет все более и более точная.

\[\dfrac{y_{k + 1} - y_k}{t_{k + 1} - t_k} = \alpha y_k\]

Тогда слева производная.
\[y’(t) = \alpha y(t) - g \cdot y\]
Как же получать такие формулы? Все-таки мы не привели ни одного аргумента, что эта формула верна…
Пусть этим занимаются физики, мы лишь будем решать используя эти формулы.

Попробуем поугадывать решения:

\[\varphi(t) = \alpha t \implies (\alpha t)’ = \alpha (\alpha t) \implies \alpha = \alpha ^ 2 t \implies t = \dfrac{1}{\alpha}\]

\[\varphi(t) = e^{\alpha t} \implies (e^{\alpha t})’ = \alpha e^{\alpha t} \implies \alpha e^{\alpha t} \equiv \alpha e^{\alpha t} \text{на} ~ \R \implies \varphi(t) = C e^{\alpha t} ~-~ \text{все решения}\]
\end{example}

\begin{problem}
Пусть дано $m(0) = 25$г, $m(30) = 42$г, $m(t_2) = 2m(0), ~ t_2 - ?$

То есть нам нужно найти точку на плоскости (здесь был рисунок), но она может быть где угодно, так что предположем, что у нас есть еще какие-то данные:
\[m’(t) = \alpha m(t) ~\&~ m(t) = C e^{\alpha t}\]
\end{problem}

\begin{proof}
[Решение]
\[25 = m(0) = C \quad 42 = m(30) = 25 e^{\alpha \cdot 30} \implies \alpha = \dfrac{1}{30} \ln \dfrac{42}{25} \implies 50 = m(t_2) = 25 e^{\dfrac{1}{30} \ln \dfrac{42}{25}} \implies t_2 \approx 40\]
\end{proof}

\subsection{Второй закон Ньютона}
\[F = ma \implies a = \dfrac{F(t, x, v)}{m} \implies \pp x  =\dfrac{F(t, x, \p x)}{m}\]
Так что дифференциальные уравнение встречаются очень часто --- мотивируйтесь их решать.

\section{Уравнения первого порядка и его решения}
\begin{definition}
Дифференциальное уравнение первого порядка --- это уравнение вида:
\[F(x, y, y’) = 0\]
\end{definition}

\begin{definition}
Функция $\varphi$, если:
\begin{enumerate}
\item $\varphi \in C^1(a, b)$
\item $F(x, \varphi(x), \varphi’(x)) \equiv 0, ~ x \in \in (a, b)$
\end{enumerate}
\end{definition}

\begin{example}
$y’ = -\dfrac{1}{x^2}$

$y = \dfrac{1}{x} + C, ~ x \in \R \setminus \{0\}$

\[y = \begin{cases}
\dfrac{1}{x} + A, x < 0 \\
\dfrac{1}{x} + B, x > 0
\end{cases}\]

Сейчас одна точка разрыва, а если их больше, то было бы больше независимых констант…
Поэтому, решениями являются функции на отрезке.
\end{example}

\begin{definition}
Интегральная кривая --- это график его решения.
\end{definition}

<!-- Опять рисунок -->

\begin{definition}
Общее множество решений для дифференциального уравнения --- это множество всех его решений.
\end{definition}

\begin{definition}
$F(x, y, c) = 0$

Общий интеграл --- это такой интеграл при некотором значении константы в решении которого, соотношение неявно задает все решения.
\end{definition}

\subsection{Уравнение в нормальной форме}
\begin{definition}\label{1.2}
Уравнение в неявной форме:
\[y’ = f(x, y)\]
\end{definition}

Для такого мы определим \textbf{область задания} --- это аналог ОДЗ.
\begin{definition}
Область задания --- это множество $\Dom f$ (domain) --- множество, где уравнение имеет смысл.
\end{definition}

\begin{example}
$y’ = -\dfrac{1}{x^2}, ~ f(x, y) = -\dfrac{1}{x^2}, ~ \Dom f = \R \setminus \{0\} \times \R$
\end{example}

\begin{problem}
$\sphericalangle y’ = x + y$. Пусть $\varphi$ --- решение. У нас есть такая связь: $\varphi’(x) = x + \varphi(x)$, в частности, в точке $(2, 3)$.

$\sphericalangle (2, 3) \quad \varphi’(2) = 2 + 3 = 5 = f(2, 3)$

То есть, если там проходит наша функция, то она проходит там под углом $\arctg 5$.

$\sphericalangle (4, 3) \quad \varphi’(4) = 4 + 3 = 7 = f(4, 3)$

Никто не мешает нам взять какую-то сетку, и в каждой точке этой сетки мы поймем, как примерно ведут себя интегральные кривые. То есть, можно не решая уравнения, можно построить такое поле и увидеть, как ведут себя интегральные кривые.

\begin{definition}
То есть, задать уравнение --- это значит увидеть, как ведет себя поле направлений.
\end{definition}
\end{problem}

Из этого геометрического смысла, мы можем сделать еще один вывод.

Возьмем какую-то точку (потом научимся их находить), посчитаем в ней угол, пойдем по этому направлению, новая точка --- новое направление, и т. д.
Чем мельче шаг, тем ближе ломаная к интегральной кривой.
\[x_{k + 1} = x_k + h\]
\[y_{k + 1} = y_k + f(x_k, y_k) h\]
\[\dfrac{\delta y_k}{\delta x_k} = f(x_k, y_k)\]

Так определяется \textbf{ломаная Эйлера}.

\subsection{Уравнение в дифференциалах}

Давайте запишем производную, как отношение дифференциалов, и перепишем  уравнение \ref{1.2}.
\[dy = f(x, y) dx\]
\[f(x, y) dx - dy = 0\]

\begin{definition}\label{1.3}
Уравнение в дифференциалах: $P(x, y) dx + Q(x, y) dy = 0$
\end{definition}

\begin{definition}
Функция $\varphi$ --- это решение \ref{1.3}, если:
\begin{itemize}
\item $\varphi \in C^1 (a, b)$
\item $P(x, \varphi(x)) + Q(x, \varphi(x)) \varphi’(x) \equiv 0, ~ x \in (a, b)$
\end{itemize}
\end{definition}

\begin{definition}
Область определения \ref{1.3} --- это множество $\Dom P \cap \Dom Q$
\end{definition}

\begin{example}
Пусть $xdx + ydy = 0 \qquad \Dom P \cap \Dom Q = \R^2$

$y = \sqrt{R^2 - x^2}, ~ x \in (-R, R) \quad xdx + \sqrt{R^2 - x^2} \cdot\left(- \dfrac{2x}{2 \sqrt{R^2 - x^2}}\right) dx = 0$
%<!-- Рисунок -->
\end{example}

В чем еще одна идея такого вида уравнения? В том, что $x$ и  $y$ здесь равноправны, то есть $x = k y$ --- это тоже \textit{конечное} решения.

\begin{definition}
Пара или вектор-функция $r(t) = (\varphi(t), \psi(t))$ --- это параметрическое решение уравнения \ref{1.3}, если:
\begin{itemize}
\item $\varphi, \psi \in C^1 (\alpha, \beta), r’(t) \neq 0, ~ \forall t \in (\alpha, \beta)$ --- второе условие, чтобы не было изломов у функции (как у модуля)
\item $P(\varphi(t), \psi(t)) \varphi’(t) + Q(\varphi(t), \psi(t)) \psi’(x) \equiv 0$
\end{itemize}
\end{definition}

\begin{example}
$xdx + ydy = 0 \implies (R \cos t, R \sin t), ~ t \in \R$ --- параметрическое решение.
\end{example}

$P(\varphi, \psi) \varphi’ + Q(\varphi, \psi) \psi’(x) \equiv 0 \implies (P, Q) \cdot (\varphi’, \psi’) = 0$

$F = (P, Q) \quad r = (\varphi, \psi) \quad F \perp r’$

<!-- Рисунок -->

И здесь у нас никакие направления не исключаются, в отличии от поля, где исключались вертикальные направления.
    
    % конец Сашиного блока
    %
    %
    %
    

    \section{Задача Коши и уравнения с разделяющимися переменными}

    \subsection{Задача Коши (ЗК)}

    \begin{definition}
        Задачей Коши или начальной задачей называется задача отыскания решения уравнения в нормальной форме $\p y = f(x,y)$, которая удовлетворяет начальному условию  $y(x_0) = y_0$

        \[\p y = f(x,y)\quad y(x_0) = y_0\]

        $(x_0, y_0)$ -- начальные данные
    \end{definition}

    Вопросы: есть ли решение и может ли их быть несколько?

    \begin{theorem}
        [Теорема о существовании для уравнений 1-го порядка]

        $G$ -- область (открытое связное множество),  $f\in C(G)$,  $(x_0, y_0)\in G \implies \exists $ решение задачи Коши в некоторой окрестности точки $x_0$
    \end{theorem}

    \begin{example}
        $\p y = f(x, y)\quad f(x, y) = \begin{cases}
            1&, y>0\\
            0, & y\leqslant 0\\
        \end{cases}$ 

\begin{figure}[!ht]
    \centering
    \incfig{legko}
    \caption{legko}
    \label{fig:legko}
\end{figure}
    \end{example}

    \begin{theorem}
        [Теорема единственности для уравнения 1-го порядка]

        $G$ -- область,  $f, \p g_y \in C(G), (x_0,y_0)\in G, \varphi_1, \varphi_2$ -- решения ЗК на $(\alpha, \beta) \implies \varphi_1 \equiv \varphi_2$ на $\left( \alpha, \beta \right) $
    \end{theorem}

    \begin{example}
        $\p y = 3\sqrt[3]{y^2} $ 

        $f(x, y) = 3\sqrt[3]{y^2} $ -- непрерывна везде, $G = \R^2$

        По теореме о существовании через любую точку проходит хотя бы одна интегральная кривая.

        $\p f_y = 3\cdot \frac{2}{3}\cdot \frac{1}{\sqrt[3]{y} } = \frac{2}{\sqrt[3]{y} }$ 

        На прямой $y = 0$ нарушаются условия теоремы об единственности, значит в этих точках могут (но не факт, что будут) проходить несколько интегральных кривых.

         \begin{gather*}
             dy = 2\sqrt[3]{y^2}dx\\
             y = (x-c)^3
        \end{gather*}

\begin{figure}[!ht]
    \centering
    \incfig{uzhas}
    \caption{uzhas}
    \label{fig:uzhas}
\end{figure}

Ответ: $y = (x-C)^3\quad C\in \R, x\in \R$

$y = 0, x\in \R$ -- особое решение. Имеются составные решения
    \end{example}

    \begin{definition}
        Решение $\varphi$ на  $(a, b)$ уравнения $\p y = f(x, y)$ называется \underline{особым}, если \[\forall x_0\in (a,b)\ \forall \varepsilon>0 \exists \varphi_1 \text{ -- решение задачи}, \p y = f(x,y)\quad y(x_0) = \varphi(x_0)\] 

        на $(\alpha,\beta)$, где  $\beta - \alpha < \varepsilon, x_0\in \left( \alpha, \beta \right) $, но $\varphi_1 \not\equiv \varphi$ на $\left( \alpha, \beta \right) $
    \end{definition}

    \subsection{Уравнения с разделяющимися переменными}

    \begin{definition}
        [Уравнения с разделёнными переменными]
        \[P(x)dx + Q(y)dy = 0\]
    \end{definition}

    \begin{theorem}
        [Общее решение уравнения с разделёнными переменными]
        $P\in C(a, b)\quad Q\in C(c, d)\quad (\alpha, \beta)\subset (a, b)$

        Тогда функция $y = \varphi(x)$ -- решение на $(\alpha,\beta) \iff :$
        \begin{enumerate}
            \item $\varphi\in C^1(\alpha, \beta)$
            \item  $\exists C\in \R$, т.ч. $\varphi$ неявно задаётся уравнением  $\int P(x)dx + \int Q(y)dy = C$
        \end{enumerate}
    \end{theorem}
    \begin{proof}
         \begin{itemize}
             \item []
             \item [$\implies $] Дано, что $\varphi$ -- решение  $\implies $ автоматически выполняется первый пункт.

                 $\sqsupset x_0\in (\alpha, \beta)$ -- прозвольно. $y_0 := \varphi(x_0)$, тогда пункт 2 запишется как:
                 \[\int_{x_0}^xP(x)dx + C_1 + \int_{y_0}^y Q(t)dt + C_2 = C\]
                 \[\int_{x_0}^xP(x)dx + \int_{y_0}^{\varphi} Q(t)dt = A\quad \forall x\in (\alpha, \beta)\]
                  
                 Пусть $t  = \varphi(\tau) \implies $ л.ч.

                 \[\int_{x_0}^x P(t)dt + \int_{x_0}^xQ\left( \varphi(\tau \right) \p \varphi(\tau)dt = \int_{x_0}^x\left( P(\tau) + Q\left( \varphi(\tau \right) \p \varphi(\tau) \right) dt \equiv 0 \text{ на } \left( \alpha, \beta \right) \]
             \item [$\impliedby $] Дано: $\varphi\in C^1(\alpha, \beta)$ и  $\int P(x)dx + \left[\int Q(y)dy \right]_{y = \varphi(x)}\equiv C$ на  $(\alpha, \beta)$

                 продиффиренцируем наше тождество (законно, потому что  $\varphi$ непрерывно дифференцируемо)

                 $P(x) + Q\left( \varphi(x) \right) \cdot \p \varphi(x)\equiv 0$ на $(\alpha, \beta)$
            
        \end{itemize}
    \end{proof}

    \begin{example}
        $xdx + ydy = 0$

         \begin{gather*}
            \int xdx + \int ydy = C\\
            x^2 + y^2 = 2C\\
            x^2 + y^2 = A
        \end{gather*}

        $A>0\quad 
        \begin{matrix}
            y = \pm \sqrt{A-x^2}&x\in (-\sqrt{A}, \sqrt{A})\\
            x = \pm \sqrt{A - y^2}&, y\in \left( -\sqrt{A}, \sqrt{A} \right)  
        \end{matrix}
        $               
    \end{example}

    \begin{definition}
        Уравнение с разделяющимися переменными:
        \[p_1(x)q_1(y)dx + p_2(x)y_2(y)dy = 0\]

        $p_2(x_0) = 0 \implies x \equiv x_0$ -- решение

        $q_1(t_0) = 0 \implies  y\equiv y_0$ -- решение

        Далее отдельно рассматриваем на каждой из областей (4 здесь):

        \[\frac{p_1(x)}{p_2(x)}dx + \frac{q_2(y)}{q_1(y)}dy = 0\]
    \end{definition}

    \begin{example}
        \[2ydx - xdy = 0\]

        $x = 0, y = 0$ -- решения. Нужно отдельно смотреть все четверти

         \begin{gather*}
            \frac{2dx}{x} = \frac{dy}{y}\\
            2\ln |x| = \ln |y| + C\\
            y = Ax^2, A>0, x>0
        \end{gather*}

        В остальных четвертях аналогично. Они все стыкуются в нуле и общее решение -- всевозможные стыковки.
\begin{figure}[!ht]
    \centering
    \incfig{gohan1}
    \caption{gohan1}
    \label{fig:gohan1}
\end{figure}
    \end{example}

    \begin{example}
        \[ydx - xdy = 0\]

        \begin{gather*}
            \int \frac{dx}{x} = \int \frac{dy}{y}  + C\\
            \ln |x| = \ln |y| + C\\
            y = Ax
        \end{gather*}

\begin{figure}[!ht]
    \centering
    \incfig{gohan2}
    \caption{gohan2}
    \label{fig:gohan2}
\end{figure}
    \end{example}

    \begin{definition}
        Два уравнения называют \underline{эквивалентными}, если они имеют одинаковую область задания и одинаковый набор интегральных кривых.
    \end{definition}
    \begin{note}
    \[p_1(x)q_1(y)dx + p_2(x)q_2(y)dy = 0\]
    не экивалентно \[\frac{p_1(x)}{p_2(x)}dx +\frac{q_1(y)}{q_2(y)}dy = 0\]
    \end{note}

    \begin{theorem}
        [Теорема о существовании и единственности для уравнений с разделёнными переменными]

        $P\in C(a, b)\quad Q\in C(c, d)$

        $(x_0, y_0)$ -- не особая точка уравнения (т.е.  $P(x_0)\neq 0, Q(y_0)\neq 0) \implies $ уравнение \[\int_{x_0}^xP(t)dt + \int_{y_0}^yQ(t)dt = 0\]

        определяет единственное решение в некоторой окрестности точки $x_0$
    \end{theorem}
\end{document}
