\documentclass{book}
%nerd stuff here
\pdfminorversion=7
\pdfsuppresswarningpagegroup=1
% Languages support
\usepackage[utf8]{inputenc}
\usepackage[T2A]{fontenc}
\usepackage[english,russian]{babel}
% Some fancy symbols
\usepackage{textcomp}
\usepackage{stmaryrd}
% Math packages
\usepackage{amsmath, amssymb, amsthm, amsfonts, mathrsfs, dsfont, mathtools}
\usepackage{cancel}
% Bold math
\usepackage{bm}
% Resizing
%\usepackage[left=2cm,right=2cm,top=2cm,bottom=2cm]{geometry}
% Optional font for not math-based subjects
%\usepackage{cmbright}

\author{Коченюк Анатолий}
\title{Вею-программирование}

\usepackage{url}
% Fancier tables and lists
\usepackage{booktabs}
\usepackage{enumitem}
% Don't indent paragraphs, leave some space between them
\usepackage{parskip}
% Hide page number when page is empty
\usepackage{emptypage}
\usepackage{subcaption}
\usepackage{multicol}
\usepackage{xcolor}
% Some shortcuts
\newcommand\N{\ensuremath{\mathbb{N}}}
\newcommand\R{\ensuremath{\mathbb{R}}}
\newcommand\Z{\ensuremath{\mathbb{Z}}}
\renewcommand\O{\ensuremath{\emptyset}}
\newcommand\Q{\ensuremath{\mathbb{Q}}}
\renewcommand\C{\ensuremath{\mathbb{C}}}
\newcommand{\p}[1]{#1^{\prime}}
\newcommand{\pp}[1]{#1^{\prime\prime}}
% Easily typeset systems of equations (French package) [like cases, but it aligns everything]
\usepackage{systeme}
\usepackage{lipsum}
% limits are put below (optional for int)
\let\svlim\lim\def\lim{\svlim\limits}
\let\svsum\sum\def\sum{\svsum\limits}
%\let\svlim\int\def\int{\svlim\limits}
% Command for short corrections
% Usage: 1+1=\correct{3}{2}
\definecolor{correct}{HTML}{009900}
\newcommand\correct[2]{\ensuremath{\:}{\color{red}{#1}}\ensuremath{\to }{\color{correct}{#2}}\ensuremath{\:}}
\newcommand\green[1]{{\color{correct}{#1}}}
% Hide parts
\newcommand\hide[1]{}
% si unitx
\usepackage{siunitx}
\sisetup{locale = FR}
% Environments
% For box around Definition, Theorem, \ldots
\usepackage{mdframed}
\mdfsetup{skipabove=1em,skipbelow=0em}
\theoremstyle{definition}
\newmdtheoremenv[nobreak=true]{definition}{Определение}
\newmdtheoremenv[nobreak=true]{theorem}{Теорема}
\newmdtheoremenv[nobreak=true]{lemma}{Лемма}
\newmdtheoremenv[nobreak=true]{problem}{Задача}
\newmdtheoremenv[nobreak=true]{property}{Свойство}
\newmdtheoremenv[nobreak=true]{statement}{Утверждение}
\newmdtheoremenv[nobreak=true]{corollary}{Следствие}
\newtheorem*{note}{Замечание}
\newtheorem*{example}{Пример}
\renewcommand\qedsymbol{$\blacksquare$}
% Fix some spacing
% http://tex.stackexchange.com/questions/22119/how-can-i-change-the-spacing-before-theorems-with-amsthm
\makeatletter
\def\thm@space@setup{%
  \thm@preskip=\parskip \thm@postskip=0pt
}
\usepackage{xifthen}
\def\testdateparts#1{\dateparts#1\relax}
\def\dateparts#1 #2 #3 #4 #5\relax{
    \marginpar{\small\textsf{\mbox{#1 #2 #3 #5}}}
}

\def\@lecture{}%
\newcommand{\lecture}[3]{
    \ifthenelse{\isempty{#3}}{%
        \def\@lecture{Lecture #1}%
    }{%
        \def\@lecture{Lecture #1: #3}%
    }%
    \subsection*{\@lecture}
    \marginpar{\small\textsf{\mbox{#2}}}
}
% Todonotes and inline notes in fancy boxes
\usepackage{todonotes}
\usepackage{tcolorbox}

% Make boxes breakable
\tcbuselibrary{breakable}
\newenvironment{correction}{\begin{tcolorbox}[
    arc=0mm,
    colback=white,
    colframe=green!60!black,
    title=Correction,
    fonttitle=\sffamily,
    breakable
]}{\end{tcolorbox}}
% These are the fancy headers
\usepackage{fancyhdr}
\pagestyle{fancy}

% LE: left even
% RO: right odd
% CE, CO: center even, center odd
% My name for when I print my lecture notes to use for an open book exam.
% \fancyhead[LE,RO]{Gilles Castel}

\fancyhead[RO,LE]{\@lecture} % Right odd,  Left even
\fancyhead[RE,LO]{}          % Right even, Left odd

\fancyfoot[RO,LE]{\thepage}  % Right odd,something additional 1  Left even
\fancyfoot[RE,LO]{}          % Right even, Left odd
\fancyfoot[C]{\leftmark}     % Center

\usepackage{import}
\usepackage{xifthen}
\usepackage{pdfpages}
\usepackage{transparent}
\newcommand{\incfig}[1]{%
    \def\svgwidth{\columnwidth}
    \import{./figures/}{#1.pdf_tex}
}
\usepackage{tikz}
\begin{document}
    \maketitle
    Http -- протокол по которому общается браузер и веб-сервер. 

    IP внутри TCP внутри HTTP внутри <> для HTTPS, такая матрёшка.

    Отец -- Тим Бернерс-Ли.

    HTTP2 бинарный, но эта бинарность от нас скрыта и мы будем думать, что он текстовый.

    Пользовательский агент -- браузер обычно, но может консолька или ещё что

    Агент посылает запрос -- сервер отвечает этими данными. Агент инициирует общение, сервер отвечает.

    TCP -- важная штука, на которой в частности базируется HTTP, там трёхфазное рукопожатие происходит, это долго.

    В HTTP1.1 придумали не разрывать соединение (keep alive) для повторных вопрос-ответов

    <картинка со слоями>

    HTTP:
    \begin{itemize}
        \item протокол предметного кровня
            |item клиент-серверный, запрос-ответ
        \item текстовый (не совсем в HTML2, чтобы оптимизировать мировой трафик)
        \item stateless -- протокол без состояния. Когда сервер обрабатывает запрос, ему не важен контекст кто откуда это прислал. Каждый запрос обрабатывается независимо от других запросов.
    \end{itemize}

    В разработке HTTP3, которое там ещё оптимизированное будет.

    Поверх HTTP строятся протоколы ещё более высокого уровня (документики типа HTML, CSS, ..)

    <картинка с url>

    \begin{definition}
        URI Universal Resource Identiier -- последовательность символов, индентифицирующая абстрактный или физический ресурс
    \end{definition}

    \begin{definition}
        Uniform Resource Locator (URL, урлы) -- стандарт записи ссылок на объекты в Интернете. Ещё есть URN  -- идентифицирует имя, например urn:isbn:5170224575
    \end{definition}

    Особенности:
    \begin{itemize}
        \item Процентное кодирование для не ASCII символов: например \%20 -- пробел. Браузеры зачастую скрывают это и показывают, например, русские буквы в url на страничке википедии
        \item схема и хост -- регистронезависимые, а всё остальное нормализуется как регистрозависимое.
        \item Используются часто относительные пути, говоряющие, что нужно сбросить путь и начать с корня.
    \end{itemize}

    глагол (GET, POST, ..) / HTTP  / 1.1 (жизнь не стоит на месте, если делаешь свой протокол, разумно хранить версию в заголовке)

    HOST -- обязательная часть заголовка www.example.com

    Общий вид:
    \begin{enumerate}
        \item METHOS URL HTTP/Version
        \item Аргументы Name: Value
        \item Пустая строка
        \item Данные
    \end{enumerate}

    Посылемые данные сжимаются, например gzip'ом, но не всегда. Обычно на сервере есть константа начиная с которой данные сжимать осмысленно.

    Cookie -- псевдосостояние, потом отдельно поговорим.

    <табличка методов>

    Некоторые заголовки:
    \begin{itemize}
        \item Общие: Cashe-Control, Connection, Sate
        \item Запрос: Accept, Accept-Encoding, Accept-Languages, Connection, Cookie, Host, Refere, User-Agent, if-None-Match
        \item Ответ: Connection, Connection-Encoding, Connection-Type, Transfer-Encoding, Set-Cookie, Location
    \end{itemize}

    Типы кодов:
    \begin{enumerate}
        \item [1xx] информационные
        \item [2xx] успех (200 OK, 206 Partial Content)
        \item [3xx] перенаправление (301 Moved Permanently, 302 Moved Temporarily, 304 Not Modified, 303 See Other, 307 Temporarily Redirect, 308 Permanent Redirect)
        \item [4xx] ошибка клиента (403 Forbidden, 404 Not Found, 405 Mthod Not Allowed)
        \item [5xx] ошибка сервера (500 Internal Server Error, 501 Not Implemented, 502 Bad Gateway, 503 Service Unavailable, 504 Gateway Timeout)
    \end{enumerate}

    Печеньки -- hashmap'чик, который хранится на стороне клиента, который параметризован именем хоста. При запросе на любую странице этого же сайта (с этим именем хоста), то сайт также посылает и эту печеньку.

    Параметры:
    \begin{itemize}
        \item  name, value
        \item expires, max-age
        \item path, domain
        \item secure, httponly, samesite
    \end{itemize}

    <cashe-control картинка>

    Отличия HTTP2:
    \begin{itemize}
        \item бинарный. Прошлый был местами бинарный, но не в заголовках, а здесь всё бинарное
    \end{itemize}

    \section{Введение в HTML/CSS}


    Будем делить элементы на блоковые (занимают всё, что дают) и инлайновые (встраиваются в строку)

    \begin{definition}
        DOM -- Document object model

        CSS -- Cascading Style Sheet
    \end{definition}
\end{document}
