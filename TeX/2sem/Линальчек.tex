\documentclass{book}
%nerd stuff here
\pdfminorversion=7
\pdfsuppresswarningpagegroup=1
% Languages support
\usepackage[utf8]{inputenc}
\usepackage[T2A]{fontenc}
\usepackage[english,russian]{babel}
% Some fancy symbols
\usepackage{textcomp}
\usepackage{stmaryrd}
% Math packages
\usepackage{amsmath, amssymb, amsthm, amsfonts, mathrsfs, dsfont, mathtools}
\usepackage{cancel}
% Bold math
\usepackage{bm}
% Resizing
%\usepackage[left=2cm,right=2cm,top=2cm,bottom=2cm]{geometry}
% Optional font for not math-based subjects
%\usepackage{cmbright}

\author{Коченюк Анатолий}
\title{Конспект по линейной алгебре\\ II семестр}

\usepackage{url}
% Fancier tables and lists
\usepackage{booktabs}
\usepackage{enumitem}
% Don't indent paragraphs, leave some space between them
\usepackage{parskip}
% Hide page number when page is empty
\usepackage{emptypage}
\usepackage{subcaption}
\usepackage{multicol}
\usepackage{xcolor}
% Some shortcuts
\newcommand\N{\ensuremath{\mathbb{N}}}
\newcommand\R{\ensuremath{\mathbb{R}}}
\newcommand\Z{\ensuremath{\mathbb{Z}}}
\renewcommand\O{\ensuremath{\emptyset}}
\newcommand\Q{\ensuremath{\mathbb{Q}}}
\renewcommand\C{\ensuremath{\mathbb{C}}}
\newcommand{\p}[1]{#1^{\prime}}
\newcommand{\pp}[1]{#1^{\prime\prime}}
\newcommand{\tl}[1]{\widetilde{#1}}
% Easily typeset systems of equations (French package) [like cases, but it aligns everything]
\usepackage{systeme}
\usepackage{lipsum}
% limits are put below (optional for int)
\let\svlim\lim\def\lim{\svlim\limits}
\let\svsum\sum\def\sum{\svsum\limits}
%\let\svlim\int\def\int{\svlim\limits}
% Command for short corrections
% Usage: 1+1=\correct{3}{2}
\definecolor{correct}{HTML}{009900}
\newcommand\correct[2]{\ensuremath{\:}{\color{red}{#1}}\ensuremath{\to }{\color{correct}{#2}}\ensuremath{\:}}
\newcommand\green[1]{{\color{correct}{#1}}}
% Hide parts
\newcommand\hide[1]{}
% si unitx
\usepackage{siunitx}
\sisetup{locale = FR}
% Environments
% For box around Definition, Theorem, \ldots
\usepackage{mdframed}
\mdfsetup{skipabove=1em,skipbelow=0em}
\theoremstyle{definition}
\newmdtheoremenv[nobreak=true]{definition}{Определение}
\newmdtheoremenv[nobreak=true]{theorem}{Теорема}
\newmdtheoremenv[nobreak=true]{lemma}{Лемма}
\newmdtheoremenv[nobreak=true]{problem}{Задача}
\newmdtheoremenv[nobreak=true]{property}{Свойство}
\newmdtheoremenv[nobreak=true]{statement}{Утверждение}
\newmdtheoremenv[nobreak=true]{corollary}{Следствие}
\newtheorem*{note}{Замечание}
\newtheorem*{example}{Пример}
\renewcommand\qedsymbol{$\blacksquare$}
% Fix some spacing
% http://tex.stackexchange.com/questions/22119/how-can-i-change-the-spacing-before-theorems-with-amsthm
\makeatletter
\def\thm@space@setup{%
  \thm@preskip=\parskip \thm@postskip=0pt
}
\usepackage{xifthen}
\def\testdateparts#1{\dateparts#1\relax}
\def\dateparts#1 #2 #3 #4 #5\relax{
    \marginpar{\small\textsf{\mbox{#1 #2 #3 #5}}}
}

\def\@lecture{}%
\newcommand{\lecture}[3]{
    \ifthenelse{\isempty{#3}}{%
        \def\@lecture{Lecture #1}%
    }{%
        \def\@lecture{Lecture #1: #3}%
    }%
    \subsection*{\@lecture}
    \marginpar{\small\textsf{\mbox{#2}}}
}
% Todonotes and inline notes in fancy boxes
\usepackage{todonotes}
\usepackage{tcolorbox}

% Make boxes breakable
\tcbuselibrary{breakable}
\newenvironment{correction}{\begin{tcolorbox}[
    arc=0mm,
    colback=white,
    colframe=green!60!black,
    title=Correction,
    fonttitle=\sffamily,
    breakable
]}{\end{tcolorbox}}
% These are the fancy headers
\usepackage{fancyhdr}
\pagestyle{fancy}

% LE: left even
% RO: right odd
% CE, CO: center even, center odd
% My name for when I print my lecture notes to use for an open book exam.
% \fancyhead[LE,RO]{Gilles Castel}

\fancyhead[RO,LE]{\@lecture} % Right odd,  Left even
\fancyhead[RE,LO]{}          % Right even, Left odd

\fancyfoot[RO,LE]{\thepage}  % Right odd,something additional 1  Left even
\fancyfoot[RE,LO]{}          % Right even, Left odd
\fancyfoot[C]{\leftmark}     % Center

\usepackage{import}
\usepackage{xifthen}
\usepackage{pdfpages}
\usepackage{transparent}
\newcommand{\incfig}[1]{%
    \def\svgwidth{\columnwidth}
    \import{./figures/}{#1.pdf_tex}
}
\usepackage{tikz}
\begin{document}
    \maketitle
    \chapter{Дополнительные главы линейной алгебры}
    \section{Полилинейная формы}

    \begin{note}
        [вспомним]
        Линейное отображение $\varphi(x+\alpha y) = \varphi(x) + \alpha\cdot \varphi(y)$
    \end{note}

    \begin{definition}
        $\sphericalangle X$ -- ЛП, $\dim X = n$

         $X^*$ -- сопряжённое к  $X$ пространство.

         \underline{Полилинейной формой} (ПЛФ) называется отображение:
          \[
         U:X\times X\times \ldots \times X\times X^*\times X^*\times \ldots \times X^* \to K
         ,\] обладающее свойством линейности по каждому аргументу.

         $\sqsupset x_1, x_2, x_3, \ldots, x_p \in X\quad y^1, y^2, \ldots, y^q\in X^*$

         $u\left( x_1, x_2, \ldots,\p x_i+\alpha \pp x_i, \ldots, x_p, y^1, y^2, \ldots, y^q \right) =\\= \left(x_1, x_2, \ldots, \p x_i, \ldots, x_p, y^1, y^2, \ldots  \right) + \alpha u\left( x_1, x_2, \ldots, \pp x_i, \ldots, x_p, y^1, y^2, \ldots, y^q \right)  $
    \end{definition}

    \begin{note}
        Пара чисел $(p,q)$ называется валентностью полилинейной формы
    \end{note}

    \begin{example}
        $\R^n\quad f:\R\to K$ -- ПЛФ $(1,0)$

        $\hat x: \R^{n*} \to  K$ -- ПЛФ(0,1)

        Скалярное произведение $u(x_1, x_2) = \left( \vec x_1, \vec x_2 \right) $ -- ПЛФ(2,0)

        Смешанное произведение $w(x_1, x_2, x_3) = (\vec x_1, \vec x_2, \vec x_3)$ -- ПЛФ(3,0)
    \end{example}

    $\sqsupset u, w$ -- две полилинейные формы валентности $(p,q)$
     \begin{definition}
        \begin{enumerate}
            \item []
            \item $u=w \iff $ \[u(x_1, \ldots, x_p, y^1, \ldots, y^q) = w(x_1, \ldots, x_p, y^1, \ldots, y^q) \quad\forall x_1\ldots x_p y^1 \ldots y^1\] 
            \item Нуль форма $\Theta\quad \Theta\left( x_1, \ldots, x_p, y^1, \ldots, y^q \right)  = 0$
            \item Суммой ПЛФ валентностей $(p,q)\quad u+v$ называется такое отображение  $\omega$, что  $\omega(x_1, \ldots, x_p, y^1, \ldots, y^1) = u\left( x_1, \ldots, x_p, y^1, \ldots, y^1 \right) + v\left( x_1, \ldots, x_p, y^1, \ldots, y^1 \right)  $
                \begin{lemma}
                    $w$ -- ПЛФ  $(p,q)$

                    $w\left( \ldots ,\p x_i+\alpha \pp x_i, \ldots \right)  = w\left( \ldots ,\p x_i ,\ldots\right) +\alpha w\left( \ldots \pp x_i \ldots \right) $
                \end{lemma}
            \item Произведением полилинейной формы на число $\lambda$ называется отображение  $\lambda u$, такое что:  \[
                    (\lambda u)\left( x_1, \ldots, x_p, y^1, \ldots, y^1 \right)  = \lambda \cdot u\left( x_1, \ldots, x_p, y^1, \ldots, y^1 \right) 
            .\] 
                    \begin{lemma}
                        $\lambda u$ -- ПЛФ  $(p,q)$
                    \end{lemma}


        \end{enumerate}
    \end{definition}

    $\sqsupset \Omega_p^q$ -- множество ПЛФ $(p,q)$

     \begin{statement}
        $\Omega_p^q$ -- ЛП
    \end{statement}


    $\sqsupset \{e_j\}$ -- базис $X\quad \sqsupset \{f^k\}$ -- базис $X^*$

    $x_1 = \sum_{j_1=1}^{n} \xi_1^{j_1} e_{j_1} = \xi_1^{j_1}e_{j_1}$. Дальше значок суммы писаться не будет (иначе помрём) (соглашение о немом суммировании).

    $x_2 = \xi_2^{j_2}e_{j_2}\quad \ldots \quad x_p = \xi^{j_p}e_{j_p}$

    $y^1 = \eta_{k_1}^1f^{k_1}\quad y_2 = \eta_{k_2}^2f^{k_2}\quad \ldots \quad y^1 = \eta_{k_q}^qf^{k_q}$ 
    \begin{align*}        
    w\left( x_1, x_2, \ldots, x_p, y^1, y^2, \ldots, y^q \right) = w\left( \xi_1^{j_1}e_{j_1}, \xi_2^{j_2}e_{j_2}, \ldots, \xi_p^{j_p}e_{j_p}, \eta_{k_1}^1f^{k_1}, \eta^2_{k_2}f^{k_2}, \ldots, \eta_{k_q}^qf^{k_q} \right) \\
    = \xi_1^{k_1}\xi_2^{j_2}\ldots\xi_p^{j_p}\eta_{k_1}^1\eta_{k_2}^2\ldots\eta_{k_q}^q \underbrace{w\left( e_{j_1}, e_{j_2}, \ldots, e_{j_p}, f^{k_1}, f^{k_2}, \ldots, f^{k_q} \right) }\limits_{\omega_{j_1j_2 \ldots j_p}^{k_1 k_2 \ldots k_q} \text{ -- тензор ПЛФ}}
    \\ = \xi_1^{k_1}\xi_2^{j_2}\ldots\xi_p^{j_p}\eta_{k_1}^1\eta_{k_2}^2\ldots\eta_{k_q}^q \omega_{j_1 j_2 \ldots j_p} ^{k_1 k_2 ... k_q}
    .\end{align*}

    \begin{lemma}
        Задание полилинейной формы эквивалентно заданию её тензора в известном базисе
        \[
            w \longleftrightarrow \omega_{i_1 i_2 \ldots i_p} ^{j_1 j_2 \ldots j_q} = \omega^{\vec j}_{\vec i}
        .\] 
    \end{lemma}
    \begin{proof}
        (выше)
    \end{proof}

    \begin{lemma}
        $v \longleftrightarrow \upsilon_{\vec i}^{\vec j}\quad w \longleftrightarrow \omega_{\vec i}^{\vec j}$

        $\implies \begin{cases}
            w+v \longleftrightarrow \upsilon_{\vec i}^{\vec j} + \omega_{\vec i}^{\vec j}\\
            \alpha v \longleftrightarrow \alpha \upsilon_{\vec i}^{\vec j}
        \end{cases}$
    \end{lemma}

     \begin{note}
         $_{t_1 t_2 \ldots t_q}^{s_1 s_2 \ldots s_p}w$ -- индексация базиса $\Omega_p^q$

     $_{t_1 t_2 \ldots t_q}^{s_1 s_2 \ldots s_p}w_{i_1 i_2 \ldots i_p}^{j_1 j_2 \ldots j_q}$

     $_{t_1 t_2 \ldots t_q}^{s_1 s_2 \ldots s_p}w\left( x_1, x_2, \ldots, x_p, y^1, y^2, \ldots, y^q \right) = \xi_1^{s_1}\xi_2^{s_2}\ldots\xi_p^{s_p}\eta_{t_1}^1\eta^2_{t_2}\ldots\eta_{t_q}^q$
    \end{note}
    \begin{note}
        $\sphericalangle _{t_1 t_2 \ldots t_q}^{s_1 s_2 \ldots s_p}w_{i_1 i_2 \ldots i_p}^{j_1 j_2 \ldots j_q} = _{t_1 t_2 \ldots t_q}^{s_1 s_2 \ldots s_p}w\left( e_{i_1}, e_{i_2}, \ldots, e_{i_p}, f^{j_1}, f^{j_2}, \ldots, f^{j_q} \right) \\
        =\delta_{i_1}^{s_1}\delta_{i_2}^{s_2}\ldots\delta_{i_p}^{s_p}\delta_{t_1}^{j_1}\delta_{t_2}^{j_2}\ldots\delta_{t_q}^{j_q}$
    \end{note}

    \begin{example}
        $\R_2^2$ 
        
        $a_1 = \begin{bmatrix} 1&0\\0&0 \end{bmatrix} \quad a_2 = \begin{bmatrix} 0&1\\0&0 \end{bmatrix} \quad a_3 = \begin{bmatrix} 0&0\\1&0 \end{bmatrix} \quad a_4 = \begin{bmatrix} 0&0\\0&1 \end{bmatrix} $

        $a_1 = ^{11}a_1\quad a_2 = ^{12}a_2\quad a_3 = ^{21}a_3\quad a_4 = ^{22}a_4$
    \end{example}

    \begin{theorem}
        Набор $\left\{ _{t_1 t_2 \ldots t_q}^{s_1 s_2 \ldots s_p} W \right\} _{s_1 s_2 \ldots s_p}^{t_1t_2 \ldots t_p} $ -- образует базис в $\Omega_q^p$
    \end{theorem}
    \begin{proof}
        \begin{itemize}
            \item []
            \item [ПН] $\sphericalangle u\in \Omega_q^p$
                \begin{align*}
                    &u\left( x_1, x_2, \ldots, x_p, y^1, y^2, \ldots, y^q \right)  = \xi_1^{i_1}\xi_2^{i_2}\ldots\xi_p^{i_p}\eta_{j_1}^1\eta_{j_2}^2\ldots\eta_{j_q}^qu_{i_1 i_2 \ldots i_p}^{j_1 j_2 \ldots j_q}=\\
                    &= _{j_1 j_2 \ldots j_q}^{i_1 i_2 \ldots i_p}w\left( x_1, x_2, \ldots, x_p, y^1, y^2, \ldots, y^q  \right) u_{i_1 i_2 \ldots i_p}^{j_1 j_2 \ldots j_q}\quad \forall x_1, x_2, \ldots, x_p, y^1, y^2, \ldots, y^q
                .\end{align*}

                $\implies u = _{j_1 j_2 \ldots j_q}^{i_1 i_2 \ldots i_p}w\cdot u_{i_1 i_2 \ldots i_p}^{j_1 j_2 \ldots j_q}$

            \item [ЛНЗ] $_{j_1 j_2 \ldots j_q}^{i_1 i_2 \ldots i_p}W\alpha_{i_1 i_2 \ldots i_p}^{j_1 j_2 \ldots j_q} = \Theta$ Посчитаем на наборе $e_{s_1}, e_{s_2}, \ldots, e_{s_p}, f^{t_1}, f^{t_2}, \ldots, f^{t_q}$

                $\delta_{s_1}^{i_1}\delta_{s_2}^{i_2}\ldots\delta_{j_1}^{t_1}\delta_{j_2}^{t_2}\ldots\delta_{j_q}^{t_q}\alpha_{i_1 i_2 \ldots i_p}^{j_1 j_2 \ldots j_q} = 0$

                $\alpha_{s_1 s_2 \ldots s_p}^{t_1 t_2 \ldots t^q} = 0\quad \forall s_1 \ldots s_p t_1 \ldots t_q \implies $ ЛНЗ (альфа 0 на всех, значит она все нули)
        \end{itemize}
    \end{proof}

    \begin{note}
        Размерность пространства полилинейных форм $\dim \Omega_q^p = n^{p+q}$
    \end{note}

    \section{Симметричные и антисимметричные ПЛФ}

    $\sphericalangle \Omega_0^p\qquad u\left( x_1, x_2, \ldots, x_p \right) $ 

    $\sphericalangle \sigma$ -- перестановка чисел от 1 до $p$.  $\sigma\left( 1, 2, \ldots, p \right)  = \left( \sigma(1), \sigma(2), \ldots, \sigma(p) \right) $

    \begin{definition}
    Полилиненйая форма $u$ называется \underline{симметричной}, если \[
            u\left( x_{\sigma(1)}, x_{\sigma(2)}, \ldots, x_{\sigma(p)} \right) = u(x_1, x_2, \ldots, x_p 
        .\]  
    \end{definition}

    \begin{lemma}
        Симметричные полилинейные формы валентности $(p,0)$ образуют подпространство $\Sigma^p$ линейного пространства $\Omega_0^p$
    \end{lemma}
    \begin{proof}
        $ \sqsupset u, v\in \Sigma^p$

        $\sphericalangle (u+v)\left( x_{\sigma(1)}, x_{\sigma(2)}, \ldots, x_{\sigma(p)} \right) = u\left( x_{\sigma(1)}, x_{\sigma(2)}, \ldots, x_{\sigma(p)} \right)  + v\left( x_{\sigma(1)}, x_{\sigma(2)}, \ldots, x_{\sigma(p)} \right) =\\=u\left( x_1, x_2, \ldots, x_p \right) + v\left( x_1, x_2, \ldots, x_p \right)  = (u+v)\left( x_1, x_2, \ldots, x_p \right)  $

        Так же с умножением на число.
    \end{proof}

    \begin{definition}
        Полилинейная форма $u$ валентности $(p,0)$ называется \underline{антисимметричной}, если:
        \[
            u\left( x_{\sigma(1)}, x_{\sigma(2)}, \ldots, x_{\sigma(p)} \right) = (-1)^{[\sigma]}u(x_1, x_2, \ldots, x_p) 
        .\]
    \end{definition}
    \begin{lemma}
        Антисимметричные полилинейные формы валентности $(p,0)$ образуют подпространство $\Lambda^p$ линейного пространства  $\Omega_0^p$
    \end{lemma}


    \begin{lemma}
        Полилинейная форма $u\in \Lambda^p \iff u = 0$ при любых двух совпадающих аргументах.
    \end{lemma}
    \begin{proof}
        \begin{itemize}
            \item []
            \item [$\implies $]
        $\sqsupset u\in \Lambda^p$ и $x_i = x_j\quad i\neq j$

        $a = \sphericalangle u\left( \ldots x_i \ldots x_j \ldots \right) = - u\left( \ldots x_j \ldots x_i \ldots \right) = -a \implies  a = 0$
    \item [$\impliedby $] Известно, что если $x_i = x_{j\neq i}$, то $u\left( \ldots x_i \ldots x_j \ldots \right)  = 0\quad \forall i, j$

        Докажем, что $u$ принадлежит  $\Lambda^p$

         $x_i = x_j = \p x_i + \pp x_i$

         $u\left( \ldots x_i \ldots x_j \ldots \right)  = u\left( \ldots \p x_i + \pp x_i \ldots \p x_i + \pp x_i\ldots \right) = u\left( \ldots \p x_i \ldots \p x_i \right) + u\left( \p x_i \ldots \pp x_i \right)  + u\left( \ldots \pp x_i \ldots \p x_i \ldots \right)  + u\left( \ldots \pp x_i \ldots \pp x_i \ldots \right)  $

         Правая часть равна 0. В левой части первое и последнее слагаемые тоже нули, а значит получам 

         \[
             u\left( \ldots \p x_i \ldots \pp x_i \right)  = -u\left( \ldots \pp x_i, \ldots \p x_i \right) 
         .\] 

         
        \end{itemize}
        
    \end{proof}

    \begin{lemma}
        Полилинейная форма $u\in \Lambda^p \iff u\left( x_1, x_2, \ldots, x_p \right) =0$ лишь только $\{x_i\}_{i=1}^p$ -- ЛЗ
    \end{lemma}
    \begin{proof}
        \begin{itemize}
            \item []
            \item [$\implies$] $\sqsupset \{x_i\}_{i=1}^p$ -- ДЗ $\implies x_k = \sum_{j\neq k}\beta^jx_j = \beta^1x_1 + \beta^2x_2 + \ldots + \beta^p x_p$

                $\sphericalangle u\left( x_1, \ldots, x_k, \ldots, x_p \right) = u\left( x_1, \ldots, \beta^1x_1+\beta^2x_2+\ldots+\beta^px_p, \ldots, x_p \right) = 0 $ 

                При раскрытии будут выносится коэффициенты и получится образ от совпадающих аргументов (хотя бы двух), который 0 по пред. лемме, а значит всё выражение, как сумма нулей, будет нулём.
            \item [$\impliedby $] $u\left( x_1, x_2, \ldots, x_p \right) = 0 $, когда $\{x_i\}_{i=1}^n$ -- ЛЗ  $\implies u\in \Lambda^p$

                $u\left( x_1, x_2, \ldots, x_p \right) = u\left( x_1 + \sum \alpha^ix_i, \ldots, x_p+\sum \alpha^ix_i \right)  = u(x_1, x_2, \ldots, x_p) + u\left( x_1, \ldots, \sum \alpha^ix_i \right) + u\left( \sum \alpha^ix_i, \ldots, x_p \right) = u\left(x_1, x_2, \ldots, x_p  \right)  + \sum_{j=2}^{p} \alpha^iu\left( x_1, \ldots, x_i \right) + \sum_{i=1}^{p-1} \alpha^iu(x_i, \ldots, x_p) $
        \end{itemize}
    \end{proof}

        \section{Практика 02.12}

        \subsection{Тензоры}

        $\omega_{i_1 i_2 \ldots i_p}^{j_1 j_2 \ldots j_q}$

        \begin{definition}
            [Соглашение об упорядочивании индексов]

            Слева направо сверху вниз: $(p,q)\quad r = p+q$ -- ранг тензора, сколько значков.

             $r = 0$ -- число  $\omega$, инвариант
            
             $r = 1$:  $a_i$ -- строчка $\begin{bmatrix} a_1&a_2&\ldots&a_n \end{bmatrix} $ $ b^j$ -- столбик  $\begin{bmatrix} b^2\\b^2\\ \vdots\\ b^n \end{bmatrix} $ 

             $r = 2$: $a_{ij}$  $b_{j}^i$ $c^{ij}$ -- первый индекс всегда строка, второй всегда столбец

             $a_{ij} \longleftrightarrow A = \begin{bmatrix} a_{11} & a_{12} & a_{13}\\ a_{21} & a_{22} & a_{23}\\ a_{31} & a_{32} & a_{33} \end{bmatrix} $ 

             $b^i_j \longleftrightarrow B = \begin{bmatrix} b_1^1 & b_2^1&b_3^1\\b_1^2 & b_2^2 & b_3^2\\ b_1^3 & b_2^3 & b_3^3 \end{bmatrix} $ 

             $r = 3:$  $a_{i j k}$  $b^i_{jk}$  $c^{ij}_k$ $d^{ijk}$

             1й -- строка, 2й -- столбец, 3й -- слой

             $a_{ijk} \longleftrightarrow A = \left[\begin{array}{c c | c c} a_{111} & a_{121} & a_{112} & a_{122}\\ a_{211} & a_{221} & a_{212} & a_{222}\end{array}\right] $ 

             \begin{example}
                 Построить тензор $\varepsilon_k^{ij} = \begin{cases}
                     -1 & (i,j,k) \text{-- чётная}\\
                     1 & (i,j,k) \text{-- чётная}\\
                 0 & (i,j,k)\text{ -- не перестановка}
                 \end{cases}$
             \end{example}

             $r = 4:$ строка, столбец, слой, сечение

             $a_{ijkl}$  $b^i_{jkl}$ $c^{ij}_{kl}$  $d^{ijk}_l$  $e^{ijkl}$ -- последний тензор типа 4,0 (число вверху, число внизу)

             $c^{ij}_{kl} \longleftrightarrow C = 
             \left[\begin{array}{c c | c c}
                 c_{11}^{11} & c_{11}^{12} & c_{12}^{11} & c_{12}^{12}\\
                 c_{11}^{21} & c_{11}^{22} & c_{12}^{21} & c_{12}^{22}\\ \hline
                 c_{21}^{11} & c_{21}^{12} & c_{22}^{11} & c_{22}^{12}\\
                 c_{21}^{21} & c_{21}^{22} & c_{22}^{21} & c_{22}^{22}\\
         \end{array}\right]$

        \end{definition}
        \begin{example}
            $c_{kl}^{ij} = \begin{cases}
                1 & i=k\neq j=l\\
            -1 & i = l \neq  j = k\\
        0 & \text{иначе}\end{cases}
            $
        \end{example}

        \subsection{Операции с тензорами}

        \begin{enumerate}
            \item Линейные операции:
                
                $\omega_{i_1 i_2 \ldots i_p}^{j_1 j_2 \ldots j_q}\qquad v_{i_1 i_2 \ldots i_p}^{j_1 j_2 \ldots j_q}$

                $u = v+\omega\quad u_{i_1 i_2 \ldots i_p}^{j_1 j_2 \ldots j_q} = (v+\omega)_{\vec i}^{\vec j} = v_{\vec i}^{\vec j} + \omega_{\vec i}^{\vec j}$ -- матричное сложение.

                $(\lambda v)_{\vec i}^{\vec j} = \lambda \cdot  v_{\vec i}^{\vec j}$
            \item Произведение:

                $u_{\vec i}^{\vec j}\quad v_{\vec s}^{\vec t}$

                $\omega = u \cdot v\quad \omega_{\vec l}^{\vec k} = u_{\vec i}^{\vec j}\cdot  v_{\vec s}^{\vec t} = \omega_{i_1 i_2 \ldots i_{p_1}s_1s_2 \ldots s_{p_2}}^{j_1 j_2 \ldots j_{q_1}t_1 t_2 \ldots t_{q_2}}$

                $\vec l = \vec j\vec t = j_1 \ldots j_{q_1}t_1 \ldots t_{q_2}$

                $\vec l = \vec i \vec s = i_1 \ldots i_{p_1}, s_1 \ldots. s_{p_2}$
        \end{enumerate}

         \begin{example}
             $a^i_j \longleftrightarrow A = \begin{bmatrix} 1 & 2\\ 3 & 4 \end{bmatrix} $

             $b_k \longleftrightarrow \begin{bmatrix} 1\\-1 \end{bmatrix} $

             $a^i_jb_k = \omega^i_{jk}$. То же самое можно записать как  $a\otimes b = \omega$

             $\omega \longleftrightarrow \left[
                 \begin{array}{c c | c c}
                     1 & 2 & -1 & -2\\ 3 & 4 & -3 & -4
                \end{array}
             \right]$

             $v = b\otimes a\quad v^i_{kj} = b_k\cdot a^i_j\longleftrightarrow V = \left[
                 \begin{array}{c c | c c}
                    1 & -1 & 2 & -2\\ 3 & -3 & 4 & -4
                \end{array}
             \right]$
        \end{example}



\begin{lemma}
    $\sqsupset \{x_{i} \}_{i=1}^p$ -- ДЗ
\end{lemma}
\begin{proof}
    $u(x_1, x_2, \ldots, x_p) = 0$

    $u\left( \alpha x_1, x_2, \ldots, x_p \right) = 0 $ 

    $u\left( \sum_{i=1}^{p-1} \alpha^ix_i, x_2, \ldots, x_p \right)  = 0$ -- равные $x_p$ и первый аргумент
\end{proof}

$\Omega_0^p$ -- хотим делать из произвольной формы симметричную

 $\sqsupset u\in \Omega_0^p$

 \begin{definition}
     $u^{(s)}\left( x_1, x_2, \ldots, x_p \right) = \frac{1}{p!}\sum_{\sigma} u\left( x_{\sigma(1)}, x_{\sigma(2)}, \ldots, x_{\sigma_{p}} \right) $ -- симметричная форма, образованная из $u$

     $u^{(s)}$ называю \underline{симметризацией}  $u$ и пишут  \[
         u^{(s)} = Sym~u
     .\] 
 \end{definition}

 \begin{note}
     $u^{(s)}\in \Sigma^p$
 \end{note}
 \begin{proof}
     $\sqsupset \tl {\sigma}$ -- другая перестановка

     $u^{(s)} \left( x_{\tl{\sigma}(1)}, \ldots, x_{\tl{\sigma}(pa)} \right)  = \frac{1}{p!}\sum_{\sigma} \left( x_{\sigma\circ \tl{\sigma}(1)}, \ldots, x_{\sigma \circ \tl{\sigma}(p)} \right) = u^{(s)}\left( x_1, x_2, \ldots, x_p \right)  $
 \end{proof}

 \begin{note}
     Деление на $p!$ нужно, чтобы выполнялось  \[
     Sym~u = u
     .\] , если $u$ уже симметричная форма
 \end{note}

 \begin{note}
     $Sym~(\alpha u+\beta v) = \alpha Sym~u + \beta Sym~v$
 \end{note}

 \begin{definition}
     \[
     u^{(a)} = \frac{1}{p!}\sum_{\sigma}(-1)^{[\sigma]} u\left( x_{\sigma(1)}, x_{\sigma(2)}, \ldots, x_{\sigma(p)} \right) 
     .\] 

     Эта операция называется \underline{антисимметризацией} или \underline{альтернированием}

     \[
         u^{(a)} = Asym~u
     .\] 
 \end{definition}
 \begin{note}
     $u^{(a)} \in \Lambda^p$
 \end{note}
 \begin{note}
     \[
         \left( \alpha u + \beta v  \right)^{(a)}  = \alpha u^{(a)} + \beta v^{(a)}
     .\] 
 \end{note}

 \begin{note}
     $Sym~Sym  = Sym$

      $Asym~Asym = Asym$

       $Sym~Asym = 0 \quad Asym~Sym = 0$
 \end{note}

 \begin{problem}
        $Omega_0^p$

     Найдём базис $\Lambda ^p$
 \end{problem}
 \begin{proof}
     $\sphericalangle \left\{ ^{s_1, s_2, \ldots, s_p} W \right\} _{\vec s}$ -- базис

     $\sphericalangle ^{s_1, s_2, \ldots, s_p}F = p!\cdot Asym \left( ^{s_1, s_2, \ldots, s_p}W \right) $

     \begin{lemma}
         Некоторые формы будут повторятся.

         $^{s_1 \ldots s_i \ldots s_j \ldots s_p}F = - ^{s_1 \ldots s_j \ldots s_i \ldots s_p} F$
     \end{lemma}

     \begin{proof}
        \begin{align*}
            ^{s1 ... s_i .. s_j \ldots s_p}F\left( x_1 \ldots x_i \ldots x_j \ldots x_p \right)  &= ^{s_1 \ldots s_j \ldots s_i \ldots s_p}F\left( x_1, \ldots, x_j, \ldots., x_i, \ldots x_p \right)\\
            &  = - ^{s_1 \ldots s_j \ldots s_p}\\
            &= - ^{s_1 \ldots s_j \ldots s_i \ldots s_p}F\left( x_1 \ldots x_i \ldots x_j \ldots x_p \right) 
        \end{align*}     
     \end{proof}

     \begin{note}
         Ненулевых $C_n^p$ штук
     \end{note}
     
     Упорядочивание $\left\{ ^{s_1 s_2 \ldots d_p} F\right\}_{1 \leqslant s_1 < s_2 < \ldots < s_p \leqslant n} $  -- ненулевой набор. Докажем, что он базис


 \end{proof}

 \begin{theorem}
     Набор $\left\{ ^{s_1 s_2 \ldots s_p}F \right\} _{1\leqslant s_1 < s_2 \ldots < s_p\leqslant  n}$ образует базис в $\Lambda^p$
 \end{theorem}
 \begin{proof}
     \begin{itemize}
         \item []
         \item [Полнота]
             $\sqsupset u\in \Lambda^p$

             \begin{align*}
                 u\left( x_1, x_2, \ldots, x_p \right) &= \xi_1^{i_1}\xi_2^{i_2}\ldots\xi_p^{i_p}u_{i_1 i_2 \ldots i_p} \\
                                                       &= ^{i_1 i_2 \ldots i_p}W\left( x_1, x_2, \ldots, x_p \right) u\left( i_1 i_2 \ldots i_p \right)  \\
                 \text{То же самое:}\quad u &= ^{i_1 i_2 \ldots i_p}W\cdot u_{i_1 i_2 \ldots i_p} \\
             .\end{align*}

             \begin{align*}
                 Asym ~ u &= Asym \left( ^{i_1 i_2 \ldots i_p}W\cdot u_{i_1 i_2 \ldots i_p} \right)  \\
                 u &= Asym\left( ^{i_1 i_2 \ldots i_p} W \right)\cdot u_{i_1 i_2 \ldots i_p}  \\
                   &= \frac{1}{p!} ~ ^{i_1 i_2 \ldots i_p}F\cdot u_{i_1 i_2 \ldots i_p} \\
                   &= \frac{1}{p!}\sum_{1\leqslant  i_1 < i_2 < \ldots < i_p \leqslant  n}\sum_{\sigma}~ ^{\sigma(i_1)\sigma(i_2) \ldots \sigma(i_n)}F \cdot u_{\sigma(i_1)\sigma(i_2) \ldots. \sigma(i_p)} \\
                   &= \frac{1}{p!}\sum_{1\leqslant i_1 < \ldots < i_p \leqslant n} \sum_{\sigma} (-1)^{[\sigma]} {}^{i_1 i_2 \ldots i_p} F \cdot  (-1)^{[\sigma]}u_{i_1 i_2 \ldots i_p} \\
                   &= \frac{1}{p!} \sum_{1 \leqslant  i_1 < \ldots < n} p! ^{i_1 i_2 \ldots i_p}F u_{i_1 i_2 \ldots i_p}\\
             .\end{align*}

             \begin{lemma}
                 $u\in \Lambda ^p \implies  \forall \sigma u_{\sigma(i_1)\sigma(i_2) \ldots \sigma(i_p) = (-1)^{[\sigma]}}u_{i_1 i_2 \ldots i_p}$

                 Тензоры это значение $u$ на  $e_{i_1} \ldots e_{i_p}$. А тогда оно выполняется просто по определению антисимметричной формы
             \end{lemma}
         \item[Линейная независимость]
             $\sphericalangle \alpha_{i_1 i_2 \ldots i_p}\quad ^{i_1 i_2 \ldots i_p}F\alpha_{i_1 i_2 \ldots i_p} = \mathds{0}$. Подействуем на $e_{s_1} e_{s_2} \ldots e_{s_p}$

             $^{i_1 i_2 \ldots i_p}F\left( e_{s_1}, e_{s_2}, \ldots, s_{s_p}  \right) \alpha_{i_1 i_2  \ldots i_p} = 0$

             $p!\left[Asym^{i_1 i_2 \ldots i_p}\right]\left( e_{s_1} e_{s_2} \ldots s_{s_p} \right) \alpha_{i_1 i_2 \ldots i_p} = 0  $ 

             $p!\cdot \frac{1}{p!}\sum_{\sigma} (-1)^{[\sigma]}{}^{i_1 i_2 \ldots i_p}W\left( e_{\sigma(s_1)}, e_{\sigma(s_2)}, \ldots, e_{\sigma(s_p)} \right) \alpha_{i_1 i_2 \ldots i_p}  = 0$ 

                 $\sum_{\sigma} (-1)^{[\sigma]}\delta_{\sigma(s_1)}^{i_1}\delta_{\sigma(s_2)}^{i_2} \ldots \delta_{\sigma(s_p)}^{i_p} \alpha_{i_1 i_2 \ldots i_p = 0}$ 

                 $\sum_{\sigma}(-1)^{[\sigma]}\alpha_{\sigma(s_1)\sigma(s_2) \ldots \sigma(s_2)} = 0$

                 $p!\alpha_{s_1 s_2 \ldots s_p} = 0 \forall  s_1 s_2 \ldots s_p \implies  \alpha = 0$, если $\alpha$ антисимметричный тензор

     \end{itemize}
 \end{proof}

 \begin{note}
     $\dim \Lambda^p = C_n^p$

      \begin{enumerate}
         \item $p = 0 \implies  C_n^0 = 1 \implies  K$
         \item $p = 1 \implies  C_n^1 = n \implies  X^*$
         \item $p = 2 \implies  C_n^2 = \frac{n(n-1)}{2}$ 
             \\ \hrule
          $C_n^p = C_n^{n-p}$ 
      \item[n] $p = n-1 \implies  C_n^{n-1} = C_n^1 = n$
      \item[n+1] $C_n^n = 1$ 

          $\sphericalangle \Lambda^n$ 

          $\left\{^{i_1 i_2 \ldots i_n}F\right\}_{1\leqslant i_1 < i_2 < \ldots < i _n \leqslant  n} = \left\{ ^{123 \ldots n}F \right\} $ 

          $\sqsupset u\in \Lambda^n \implies  \exists \alpha \quad u = \alpha^{1 2 3 \ldots n}F$

          $\sphericalangle ^{1 2 3\ldots n}F\left( x_1, x_2, \ldots, x_n \right)  = p!\cdot \left[Asym^{1 2 3\ldots n}W\right](x_1, x_2, \ldots, x_n) = \sum_{\sigma} (-1)^{[\sigma]_{1 2 3 \ldots n}}W\left( x_{\sigma(1)} x_{\sigma(2)} \ldots x_{\sigma(n)} \right)  = (-1)^{[\sigma]} \xi_{\sigma(1)}^{1} \xi_{\sigma(2)}^{2} \ldots \xi_{\sigma(n)}^{n} \overset{\triangle}= \det\{x_i\}$ 

          \begin{lemma}
              $\forall u\in \Lambda^n\quad u = \alpha\left( ^{123 \ldots n}F \right) $
          \end{lemma}
          \begin{proof}
              \begin{align*}
                  u\left( x_1, x_2, \ldots, x_n \right) &=  \xi_1^{i_1}\xi_2^{i_2} \ldots \xi_n^{i_n} u_{i_1 i_2 \ldots i_n}\\
                  &= ^{i_1, i_2, \ldots, i_n}W\left( x_1, x_2, \ldots, x_n \right) u_{i_1 i_2\ldots, i_n}\\
                  & = ^{1 2 3 \ldots n}F(x_1, x_2, \ldots, x_n)\underbrace{u_{1 2 \ldots n}}\limits_{\alpha}
              \end{align*}                        
          \end{proof}
  \end{enumerate}

  \section{Произведение полилинейных форм}

$\sqsupset \Omega_p^q$

\begin{definition}
    $u\in \Omega_{q_1}^{p_1}, v\in \Omega_{q_2}^{p_2}$

    $\sphericalangle \omega \left( x_1, x_2, \ldots, x_{p_1}, x_{p_1+1}, \ldots, x_{p_1+p_2}, y^1, \ldots, y^{q_1}, y^{q_1+1}, \ldots, y^{q_1+q_2} \right) =\\ = u\left( x_1, x_2, \ldots, x_{p_1}, y^1, y^2, \ldots, y^{q_1} \right)  \cdot v\left( x_{p_1+1}, x_{p_1+2}, \ldots, x_{p_1+p_2}, y^{q_1+1}, \ldots, y^{q_1+q_2} \right) $ 

    Такая форма называется консолидированной формой $u$ и  $v$

    $u^{i_1 i_2 \ldots i_{p_1}}_{j_1 j_2 \ldots j_{q_1}} \cdot v_{t_1 t_2 \ldots t_{q_2}}^{s_1 s_2 \ldots s_{p_2}} = \omega^{i_1 \ldots i_p, s_1, \ldots, s_{p_2}}_{j_1 \ldots j_{q_1} t_1, \ldots, t_{q_2}}$
\end{definition}

\begin{note}
    $\omega$ -- ПЛФ  $(p_1+p_2, q^1+ q^2)$

    $\omega = u\cdot v \subseteq \Omega_{q_1+q_2}^{p_1+p_1}$

    $\sphericalangle \Omega = \dotplus \sum_{i, j} \Omega_{q_j}^{p_i}$ -- линейное пространство

    $\left( \Omega, +, \cdot \lambda, \cdot  \right) $ Новое умножение называется внешним
\end{note}

\begin{property}
    \begin{enumerate}
        \item $u\cdot (v\cdot w) = (u*v)\cdot w$
        \item $u\cdot v \neq  v \cdot  u$
        \item $u\cdot (v+w) = u\cdot v+u\cdot w$
        \item $\mathds{0}\quad u\cdot \mathds{0} = \mathds{0}$ -- получившийся нооль из бОльшего пространства
        \item $u(\alpha v) = (\alpha u)\cdot v$
    \end{enumerate}
\end{property}

\begin{definition}
    $\Omega$ -- внешняя алгебра полилинейных форм
\end{definition}
 \end{note}

 \section{Практика №2}
 \subsection{Свёртки}
\begin{example}
    $\omega^{j}_i \sim \begin{pmatrix} 1 & 2 & 3 \\ 8 & 7 & 5\\ 1 & -1 & 9 \end{pmatrix} $

    $w_i^i = \sum_i = \omega_1^1 +\omega_2^2 + \omega_3^3$
\end{example}

\begin{example}
    $w^{ij}_k \sim  \left( 
        \begin{array}{c c| c c}
            1 & 2 & 8 & 9\\ 5 & -1 & 10 & 3
        \end{array}
    \right) $

    $w_i^{ij} = \alpha^j\qquad \alpha^0 = 1 + 10 = 11\quad \alpha^1 = 2+(-3) = -1$

    $\omega^{ij}_i = \begin{pmatrix} 11\\-1 \end{pmatrix} $
\end{example}

\begin{example}
    $\omega^{ij}_{kl} \sim \left( 
        \begin{array}{c c| c c}
            3 & -1 & 4 & 7\\ -8 & 1 & -3 & 11\\ \hline -3 & 4 & 13 & 17\\ 6 & 5 & 19 & 23\\
        \end{array}
    \right) $

    $\omega_{ki}^{ij} = \alpha_k^j \sim \begin{pmatrix} 0 & 10\\ 16 & 27 \end{pmatrix} $

    $\omega^{ij}_{ji} = \sum_j\sum_i\omega^{ij}_{ji} = \sum_k \alpha_k^k = \alpha_0^0 + \alpha_1^1 = 27$
\end{example}

\begin{note}
    Сложную свёртку можно считать как последовательность единичных
\end{note}

\subsection{Транспонирование}

$\omega^i_{jk} = \left( 
    \begin{array}{c c c| c c c| c c c}
        1 & 2 & 3 & 7 & 8 & 9 & 4 & 5 & 6\\
        -3 & -2 & -1 & 9 & 8 & 7 & -7 & 11 & -13\\
        2 & 19 & 17 & 14 & 12 & 9 & 21 & 17 & -1\\
    \end{array}
\right) $

$\psi_{jk}^i = \omega_{kl}^i$

$\psi_{jk}^{i} \sim \left( 
    \begin{array}{c c c | c c c| c c c}
        1 & 7 & 4 & 2 & 8 & 5 & 3 & 9 & 6\\
        -3 & 9 & -7 & -2 & 8 & 11 & -1 & 7 & -13\\
        2 & 14 & 21 & 19 & 12 & 17 & 17 & 9 & -1\\
\end{array}
\right) $ 

$\omega^{ij}_{kl} \sim \left( 
    \begin{array}{c c | c c }
        14 & 1 & 1 & 3\\
        -8 & 2 & -7 & -1\\
        18 & 16 & 9 & 11\\
        -14 & -3 & 17 & 19\\
\end{array}
\right) $ 


$\omega^{ij}_{kl} \sim \left( 
    \begin{array}{c c | c c}
        14 & & 18 & \\
           &&& \\
        1 & 3 & 9 &\\
          &&17&\\
\end{array}
\right) $


\subsection{Свёртка и тензорное произведение}

$a^{ij} \sim \begin{pmatrix} 8 & 9 & 1\\7 & -5 & 4\\ 1 & 1 & 1\end{pmatrix} $ 

$b^k_l \sim \begin{pmatrix} 7 & -1 ^ -3\\ 8 & 4 & 5\\ 11 & -9 & 1 \end{pmatrix} $ 

$a \otimes b = \omega^{ijk}_l \implies  \omega^{ijk}_j = \beta^{ik}$

$\beta^{ik} = a^{ij}b^k_l = \begin{pmatrix} 44 & & \\ & 56 & \\ & & &\\ \end{pmatrix} $

$\beta^{00} = a^{00}b^0_0 + a^{01}b^0_1 + a^{02}b^0_2 = $


$a^{ij}_k \sim \left( 
    \begin{array}{c c | c c}
     2 & 1 & -1 & -7\\
     2 & 2 & 8 & 11\\
\end{array}
\right) $ 

$b_{m, n} \sim \begin{pmatrix} 3 & -3\\ 8 & -1 \end{pmatrix} $ 

$a\otimes b = \omega^{ij}_{kmn} \implies  \omega^{ij}_{kji} = \beta_k \sim \begin{pmatrix} 9\\-94 \end{pmatrix} $

$\omega\in \Omega_0^2\quad \omega(x, y)\in \R\quad x, y\in X$

$\omega \sim  a_{ij}\quad x\sim \xi^k\quad y \sim \eta^l$

$\omega(x, y) = a_{ij}\xi^i\eta^j = (a\otimes x \otimes y)^{ij}_{ij} $

\subsection{Симметризация и асимметризация тензоров}

$\omega_{ij} \sim \begin{pmatrix} 1 & 3 & 7\\-1 & 8 & 4\\ 3 & 2 & -1 \end{pmatrix} $

$sym(\omega_{i_1 , \ldots, i_p}) = a_{j_1 \ldots j_p} = \sum_{\sigma}\omega_{i_{\sigma(1)}, \ldots, i_{\sigma(p)}}$

$a_{ij} = w_{(ij)} \sim \begin{pmatrix} 1 & 1 & 5\\ 1& 8 & 3\\5 &3 & -1\\ \end{pmatrix} $ 

$a_{ij} = \frac{1}{2!}(\omega_{ij} + \omega_{ji})$

$\omega_{ijk} \sim \left( 
    \begin{array}{c c c | c c c| c c c}
        1 & 2 & 3 & -7 & 8 & 1 & 9 & 3 & 13\\
        3 & -4 & 5 & 11 & -7 & 13 & 1 & 4 & 2\\
        8 & 9 & 7 & 5 & 6 & 11 & -7 & 8 & -1\\
\end{array}
\right) $


$a_{ijk} = \frac{1}{6}\left( \ldots \right) $ 

$a_{ijk} \sim \left( 
    \begin{array} {c c c| c c c| c c c}
        1 & -\frac{2}{3} & \frac{20}{3} & -\frac{2}{3} & 5 & 4 & \frac{20}{3} &4 & \frac{13}{3}\\
        -\frac{2}{3} & 5 & 4 & 5 & -7 & \frac{23}{3} & 4& \frac{23}{3} & 7 \\
        \frac{20}{3} &4 & \frac{13}{3} &4 &\frac{23}{3} &7& \frac{13}{3} &7 & -1\\
    
\end{array}
\right) $
















\end{document}
