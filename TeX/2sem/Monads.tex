\documentclass{book}
%nerd stuff here
\pdfminorversion=7
\pdfsuppresswarningpagegroup=1
% Languages support
\usepackage[utf8]{inputenc}
\usepackage[T2A]{fontenc}
\usepackage[english,russian]{babel}
% Some fancy symbols
\usepackage{textcomp}
\usepackage{stmaryrd}
% Math packages
\usepackage{amsmath, amssymb, amsthm, amsfonts, mathrsfs, dsfont, mathtools}
\usepackage{cancel}
% Bold math
\usepackage{bm}
% Resizing
%\usepackage[left=2cm,right=2cm,top=2cm,bottom=2cm]{geometry}
% Optional font for not math-based subjects
%\usepackage{cmbright}

\author{Коченюк Анатолий}
\title{От типов до свободных монад}

\usepackage{url}
% Fancier tables and lists
\usepackage{booktabs}
\usepackage{enumitem}
% Don't indent paragraphs, leave some space between them
\usepackage{parskip}
% Hide page number when page is empty
\usepackage{emptypage}
\usepackage{subcaption}
\usepackage{multicol}
\usepackage{xcolor}
% Some shortcuts
\DeclareMathOperator{\Obj}{Obj}
\DeclareMathOperator{\Morph}{Morph}
\newcommand\N{\ensuremath{\mathbb{N}}}
\newcommand\R{\ensuremath{\mathbb{R}}}
\newcommand\Z{\ensuremath{\mathbb{Z}}}
\renewcommand\O{\ensuremath{\emptyset}}
\newcommand\Q{\ensuremath{\mathbb{Q}}}
\renewcommand\C{\ensuremath{\mathbb{C}}}
\newcommand{\p}[1]{#1^{\prime}}
\newcommand{\pp}[1]{#1^{\prime\prime}}
% Easily typeset systems of equations (French package) [like cases, but it aligns everything]
\usepackage{systeme}
\usepackage{lipsum}
% limits are put below (optional for int)
\let\svlim\lim\def\lim{\svlim\limits}
\let\svsum\sum\def\sum{\svsum\limits}
%\let\svlim\int\def\int{\svlim\limits}
% Command for short corrections
% Usage: 1+1=\correct{3}{2}
\definecolor{correct}{HTML}{009900}
\newcommand\correct[2]{\ensuremath{\:}{\color{red}{#1}}\ensuremath{\to }{\color{correct}{#2}}\ensuremath{\:}}
\newcommand\green[1]{{\color{correct}{#1}}}
% Hide parts
\newcommand\hide[1]{}
% si unitx
\usepackage{siunitx}
\sisetup{locale = FR}
% Environments
% For box around Definition, Theorem, \ldots
\usepackage{mdframed}
\mdfsetup{skipabove=1em,skipbelow=0em}
\theoremstyle{definition}
\newmdtheoremenv[nobreak=true]{definition}{Определение}
\newmdtheoremenv[nobreak=true]{theorem}{Теорема}
\newmdtheoremenv[nobreak=true]{lemma}{Лемма}
\newmdtheoremenv[nobreak=true]{problem}{Задача}
\newmdtheoremenv[nobreak=true]{property}{Свойство}
\newmdtheoremenv[nobreak=true]{statement}{Утверждение}
\newmdtheoremenv[nobreak=true]{corollary}{Следствие}
\newtheorem*{note}{Замечание}
\newtheorem*{example}{Пример}
\renewcommand\qedsymbol{$\blacksquare$}
% Fix some spacing
% http://tex.stackexchange.com/questions/22119/how-can-i-change-the-spacing-before-theorems-with-amsthm
\makeatletter
\def\thm@space@setup{%
  \thm@preskip=\parskip \thm@postskip=0pt
}
\usepackage{xifthen}
\def\testdateparts#1{\dateparts#1\relax}
\def\dateparts#1 #2 #3 #4 #5\relax{
    \marginpar{\small\textsf{\mbox{#1 #2 #3 #5}}}
}

\def\@lecture{}%
\newcommand{\lecture}[3]{
    \ifthenelse{\isempty{#3}}{%
        \def\@lecture{Lecture #1}%
    }{%
        \def\@lecture{Lecture #1: #3}%
    }%
    \subsection*{\@lecture}
    \marginpar{\small\textsf{\mbox{#2}}}
}
% Todonotes and inline notes in fancy boxes
\usepackage{todonotes}
\usepackage{tcolorbox}

% Make boxes breakable
\tcbuselibrary{breakable}
\newenvironment{correction}{\begin{tcolorbox}[
    arc=0mm,
    colback=white,
    colframe=green!60!black,
    title=Correction,
    fonttitle=\sffamily,
    breakable
]}{\end{tcolorbox}}
% These are the fancy headers
\usepackage{fancyhdr}
\pagestyle{fancy}

% LE: left even
% RO: right odd
% CE, CO: center even, center odd
% My name for when I print my lecture notes to use for an open book exam.
% \fancyhead[LE,RO]{Gilles Castel}

\fancyhead[RO,LE]{\@lecture} % Right odd,  Left even
\fancyhead[RE,LO]{}          % Right even, Left odd

\fancyfoot[RO,LE]{\thepage}  % Right odd,something additional 1  Left even
\fancyfoot[RE,LO]{}          % Right even, Left odd
\fancyfoot[C]{\leftmark}     % Center

\usepackage{import}
\usepackage{xifthen}
\usepackage{pdfpages}
\usepackage{transparent}
\newcommand{\incfig}[1]{%
    \def\svgwidth{\columnwidth}
    \import{./figures/}{#1.pdf_tex}
}
\usepackage{tikz}
\usepackage{listings}
\begin{document}
    \maketitle
    \section{Функции}

    $f(x) = y$

    $f(x) = y_1, \ldots, y_n$ -- многозначная функция

    У нас функция: соответствие одному $y$-ку

    Зачем нужны? Много чего ими описывается. А теперь мы хотим поменять способ описания самих функций. (страшное слово -- функторизация)

    $f(x) = y \quad x\in X_f\quad y\in Y_f$

    \begin{lstlisting}
        function int mod2(int a) {
            return a & 1;
        }   
    \end{lstlisting}

    $X_f \to Y_f$ -- наша функция стала стрелочкой и это то, чем занимает теория категорий -- стрелочки)
    
    Тип $ \leftrightarrow$ Множество

    Что можем делать с множествами:  $\cap\quad \cup\quad add\quad \in?\quad \subseteq \quad \setminus \quad remove$ -- с последней есть некоторые проблемы в разных теориях, но сейчас не об этом.

    Что можем делать с типами:
    \begin{itemize}
        \item объединять да
        \item пересекать неочевидно непонятно
        \item добавлять элемент -- объединять с синглтоном
        \item принадлежность есть
        \item включённость есть
        \item с вычитанием проблемы
    \end{itemize}

    Воспомиания о топологии: Топология -- множество открытых множеств. Если $X, Y$ -- открытые, то  $X\subseteq  X\cup Y\quad Y\subseteq X\cup Y$. Этого про топологию нам должно хватить

    \section{Типы}

    \begin{lstlisting}
        class A<T,U> {
            T a;
            U b;
        }
    \end{lstlisting}

    Если значений $T$ -- 5, а  $U$ -- 7, то всего разных  $A$ Бывает 35. Вжух, мы ввели умножение

    Если ввести union(U, T), то это будет сложение

     $f: U_7 \to T_5$ -- $5^7$ -- возведение в степень

     $x+y = (1 + (x-1)) + y = \ldots = \left( 1+1+\ldots+1 \right) +y$

     $x\cdot y = (1+(x-1))\cdot y = y + (x-1)\cdot y = \ldots = y + \ldots + y$

     $x^y\quad 0^y = 0\quad x^{1+(y-1)} = x\cdot x^{y-1}$

    Рассмотрим наш класс выше. $\frac{\partial TU}{\partial U} = T$ 

    \section{Схема существования}

    Хотим, чтобы функция работали и на типах и на множествах. И отдельно, чтобы в качестве функция были наши стрелочки (которые будут называeтся категориями)

    Хотим от стрелок ассоциативности $f\circ (g\circ h) = (f\circ g)\circ h$

    $f\left( X_f \right)  = Y_f$

    \begin{lstlisting}
        func T id(T t) {
            return t
        }

        id: T -> T
        id: A -> A
    \end{lstlisting}

    Разные ли эти два id? Реализация одна, вообще разные

    \begin{lstlisting}
        A a = new A();
        A b = new A();
        id(a) != id(b)
    \end{lstlisting}

    Хотим, ли мы, чтобы они были равны. Этим вопросом в частности задаётся теория категорий


    Obj Объекты: Будем обозначать их буковками A, B, C, $\ldots$, а остальное выводить сами.

    Morph Морфизмы: $\to \quad A\to B, \ldots, f, g, \ldots$

    Закон композиции: $x, y\in Morph\quad x_{AB}, y_{BC}\quad (y\circ x)\in Morph$

    Если у нас есть две категории, и между ними есть соответствие, то мы называем его функтором

    $\C_1\quad \C_2\quad \gamma\mathscr{F}:$
    \begin{enumerate}
        \item $\forall a\in \Obj \C_1\ \exists f(a) = a\in \Obj \C_2$
        \item $\forall m\in \Morph \C_1\ \exists  f(m) = \p m$
        \item $f\left( m\left( a \right)  \right)  = \p m\left( \p a \right) , \p m = f(m), \p a = f(a)$
        \item $\forall m_1, m_2\in \Morph \C_1\quad f\left( m_1 \right) \circ f\left(m_2 \right) = \p m_1\circ \p m_2$
    \end{enumerate}
    Объект -- первого порядка, Морфизмы -- второго, Морфизмы морфизмов -- третьего, ...

    Категория какого порядка?
    \begin{itemize}
        \item  Первого, как объект
        \item 2.5, 1.5, ..
        \item Категория не объект, а что-то ещё (более страшная терминология, юхху)
    \end{itemize}

    Если есть стрелки в обе стороны, то $(f_{BA}\circ f_{AB} )(A) = id_A(A) = A$

    $A[k]\quad arr\quad f(A) = B\quad Aarr \to Barr$

    $B[k]\quad map(f, arr)$

    \begin{definition}
        Полная категория -- в которой стоят все стрелки, которые могут стоять
    \end{definition}

    Ввести петли так просто, что давайте их требовать
\end{document}
